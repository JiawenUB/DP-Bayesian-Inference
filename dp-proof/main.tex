%%%%%%%%%%%%%%%%%%%%%%%%%%%%%%%%%%%%%%%%%
% University/School Laboratory Report
% LaTeX Template
% Version 3.1 (25/3/14)
%
% This template has been downloaded from:
% http://www.LaTeXTemplates.com
%
% Original author:
% Linux and Unix Users Group at Virginia Tech Wiki 
% (https://vtluug.org/wiki/Example_LaTeX_chem_lab_report)
%
% License:
% CC BY-NC-SA 3.0 (http://creativecommons.org/licenses/by-nc-sa/3.0/)
%
%%%%%%%%%%%%%%%%%%%%%%%%%%%%%%%%%%%%%%%%%

%----------------------------------------------------------------------------------------
%	PACKAGES AND DOCUMENT CONFIGURATIONS
%----------------------------------------------------------------------------------------

\documentclass{article}
\usepackage{geometry}
\geometry{left=2.5cm,right=2.5cm,top=2.5cm,bottom=3.5cm}
\usepackage[version=3]{mhchem} % Package for chemical equation typesetting
\usepackage{siunitx} % Provides the \SI{}{} and \si{} command for typesetting SI units
\usepackage{graphicx} % Required for the inclusion of images
\usepackage{natbib} % Required to change bibliography style to APA
\usepackage{amsmath} % Required for some math elements 
\usepackage{datetime}
\setlength\parindent{0pt} % Removes all indentation from paragraphs

\renewcommand{\labelenumi}{\alph{enumi}.} % Make numbering in the enumerate environment by letter rather than number (e.g. section 6)

%\usepackage{times} % Uncomment to use the Times New Roman font

%----------------------------------------------------------------------------------------
%	DOCUMENT INFORMATION
%----------------------------------------------------------------------------------------

\title{\textbf{Proof of DP - Bayesian Inference}\\
% \begin{large}
% \textbf{Probability Distributions and Bayesian Networks}% Title
% \end{large}
}
\author{Jiawen \textsc{Liu}} % Author name

% \date{\currenttime\ \today} % Date for the report

\begin{document}

\maketitle % Insert the title, author and date

% \begin{center}
% \begin{tabular}{l r}
% Date Performed: & January 1, 2012 \\ % Date the experiment was performed
% %Partners: & James Smith \\ % Partner names
% %& Mary Smith \\
% Instructor: & Professor Sargur N. Srihari % Instructor/supervisor
% \end{tabular}
% \end{center}

% If you wish to include an abstract, uncomment the lines below
% \begin{abstract}
% Abstract text
% \end{abstract}

%----------------------------------------------------------------------------------------
%	SECTION 1
%----------------------------------------------------------------------------------------
If we want the posterior distribution from the Bayesian inference is $\epsilon$ - differential privacy, this mechanism should satisfy:
\begin{equation*}
\frac{Pr[BayesInfer_{DP}(d_1) \in S]}{Pr[BayesInfer_{DP}(d_2) \in S]} \leq e^{\epsilon},
\end{equation*}
where $BayesInfer_{DP}$ is a differentially private Bayesian inference mechanism: $d \rightarrow posterior\ distribution$. $d_1$, $d_2$ is a pair of adjacent observed data sets. Since in our model, this inference is based on a bias $p$ with beta prior distribution: $p \sim Beta(\alpha_0, \beta_0)$ and an observed data set $d \sim B(n,p)$, which result in a posterior distribution on $p \sim Beta(\alpha_0 + k, \beta_0 + l)$ where $k$ and $l$ is the number of $1$ and $0$ in $d$. As a result, the requirement above can be rewrote as:
\begin{equation*}
\frac{Pr[Beta(\alpha_0 + k^*_1, \beta_0 + l^*_1) \in S]}{Pr[Beta(\alpha_0 + k^*_2, \beta_0 + l^*_2) \in S]} \leq e^{\epsilon},
\end{equation*}
where $k^*_1$ and $l^*_1$ is the number of $1$ and $0$ in $d_1$ after protection, the same with $k^*_2$ and $l^*_2$.  $S\subseteq B = \{Beta(a,b) | a,b \in N\}$. The equation above is equivalent to:
\begin{equation*}
\frac{Pr[\langle \alpha_0 + k^*_1, \beta_0 + l^*_1 \rangle = S]}{Pr[ \langle \alpha_0 + k^*_2, \beta_0 + l^*_2 \rangle = S]} \leq e^{\epsilon}.
\end{equation*}

Under Laplace mechanism, the sensitivity of $BayesInfer_{DP}: d \rightarrow k$, $d \rightarrow l$ are both $1$, we have $k^* = k + Lap(\frac{1}{\frac{\epsilon}{2}}) = k + Lap(\frac{2}{\epsilon})$, and $l^* = l + Lap(\frac{2}{\epsilon})$. When $\langle \alpha_0 + k^*, \beta_0 + l^* \rangle = S$, since $S = Beta(s_1,s_2) = \langle s_1, s_2 \rangle $, we have $ (s_1 - \alpha_0  - k) \sim Lap(\frac{2}{\epsilon}), (s_2 - \beta_0 - l) \sim Lap(\frac{2}{\epsilon})$. Then,
\begin{equation*}
\frac{Pr[\langle \alpha_0 + k^*_1, \beta_0 + l^*_1 \rangle = S]}{Pr[\langle \alpha_0 + k^*_2, \beta_0 + l^*_2 \rangle = S]} = \frac{exp(-\frac{(s_1 - \alpha_0  - k_1)}{\frac{2}{\epsilon}})) exp(-\frac{(s_2 - \beta_0  - l_1)}{\frac{2}{\epsilon}}))}{exp(-\frac{(s_1 - \alpha_0  - k_2)}{\frac{2}{\epsilon}})) exp(-\frac{(s_2 - \beta_0  - l_2)}{\frac{2}{\epsilon}}))} = exp(\epsilon)
\end{equation*}
i.e.
\begin{equation*}
\frac{Pr[BayesInfer_{DP}(d_1) \in S]}{Pr[BayesInfer_{DP}(d_2) \in S]} \leq e^{\epsilon},
\end{equation*}

%----------------------------------------------------------------------------------------
%	SECTION 6
%----------------------------------------------------------------------------------------

%----------------------------------------------------------------------------------------
%	BIBLIOGRAPHY
%----------------------------------------------------------------------------------------

% \bibliographystyle{apalike}

% \bibliography{dp-proof}

%----------------------------------------------------------------------------------------


\end{document}