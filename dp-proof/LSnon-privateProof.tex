%%%%%%%%%%%%%%%%%%%%%%%%%%%%%%%%%%%%%%%%%
% University/School Laboratory Report
% LaTeX Template
% Version 3.1 (25/3/14)
%
% This template has been downloaded from:
% http://www.LaTeXTemplates.com
%
% Original author:
% Linux and Unix Users Group at Virginia Tech Wiki 
% (https://vtluug.org/wiki/Example_LaTeX_chem_lab_report)
%
% License:
% CC BY-NC-SA 3.0 (http://creativecommons.org/licenses/by-nc-sa/3.0/)
%
%%%%%%%%%%%%%%%%%%%%%%%%%%%%%%%%%%%%%%%%%

%----------------------------------------------------------------------------------------
%	PACKAGES AND DOCUMENT CONFIGURATIONS
%----------------------------------------------------------------------------------------

\documentclass{article}
\usepackage{geometry}
\usepackage{amssymb}
\geometry{left=2.5cm,right=2.5cm,top=2.5cm,bottom=3.5cm}
\usepackage[version=3]{mhchem} % Package for chemical equation typesetting
\usepackage{siunitx} % Provides the \SI{}{} and \si{} command for typesetting SI units
\usepackage{graphicx} % Required for the inclusion of images
\usepackage{natbib} % Required to change bibliography style to APA
\usepackage{amsmath} % Required for some math elements 
\usepackage{datetime}
\usepackage[usenames, dvipsnames]{color}
\setlength\parindent{0pt} % Removes all indentation from paragraphs
\newcommand{\mg}[1]{\textcolor[rgb]{.90,0.00,0.00}{[MG: #1]}}

\renewcommand{\labelenumi}{\alph{enumi}.} % Make numbering in the enumerate environment by letter rather than number (e.g. section 6)

%\usepackage{times} % Uncomment to use the Times New Roman font

%----------------------------------------------------------------------------------------
%	DOCUMENT INFORMATION
%----------------------------------------------------------------------------------------

\title{\textbf{Proof: Non-private Local Sensitivity}\\
% \begin{large}
% \textbf{Probability Distributions and Bayesian Networks}% Title
% \end{large}
}
% \author{Jiawen \textsc{Liu}} % Author name

% \date{\currenttime\ \today} % Date for the report

\begin{document}

\maketitle % Insert the title, author and date

% \begin{center}
% \begin{tabular}{l r}
% Date Performed: & January 1, 2012 \\ % Date the experiment was performed
% %Partners: & James Smith \\ % Partner names
% %& Mary Smith \\
% Instructor: & Professor Sargur N. Srihari % Instructor/supervisor
% \end{tabular}
% \end{center}

% If you wish to include an abstract, uncomment the lines below
% \begin{abstract}
% Abstract text
% \end{abstract}

%----------------------------------------------------------------------------------------
%	SECTION 1
%----------------------------------------------------------------------------------------

For a query $Q$ over databases: $Q: X^n \rightarrow \mathbb{R}$\mg{it
was wrongly $Q: x \rightarrow \mathbb{R}$ }, we
have the definition of local sensitivity at $x\in X^n$:
\begin{equation*}
\Delta_{local} Q(x) = \max_{\{x'\ adj\ x\}}| Q(x)- Q(x')|.
\end{equation*}

\section{Laplace Mechanism under Local Sensitivity}

The Laplace noised query $Q^{*}$ scaled to local sensitivity is:

\begin{equation*}
Q^{*}(x) = Q(x) + Lap(\frac{\Delta_{local} Q(x)}{\epsilon}).
\end{equation*}

Suppose we have two adjacent data bases $x,y$. The privacy loss of this noised query $Q$ can be analyzed as:

\begin{equation*}
\begin{split}
privacy\ loss = \frac{P[Q^{*}(x) = R]}{P[Q^{*}(y) = R]}
& = \frac{P[Q(x) + Lap(\frac{\Delta_{local} Q(x)}{\epsilon}) = R]}{P[Q(y) + Lap(\frac{\Delta_{local} Q(y)}{\epsilon}) = R]}\\
& = \frac{P[Lap(\frac{\Delta_{local} Q(x)}{\epsilon}) = R - Q(x)]}{P[Lap(\frac{\Delta_{local} Q(y)}{\epsilon}) = R - Q(y)]}\\
\end{split}
\end{equation*}
We can then reason about the pdf of Laplace:
\begin{equation*}
\begin{split}
\frac{exp(-\frac{|R - Q(x)|}{\frac{\Delta_{local} Q(x)}{\epsilon}})}{exp(-\frac{|R - Q(y)|}{\frac{\Delta_{local} Q(y)}{\epsilon}})}
& = exp(\frac{\epsilon |R - Q(y)|}{\Delta_{local} Q(y)} - \frac{\epsilon |R - Q(x)|}{\Delta_{local} Q(x)}).
\end{split}
\end{equation*}

We want to show that when $\Delta_{local} Q(x) \neq \Delta_{local}
Q(y) $ we do not have differential privacy. Without loss of generality
we can consider the case $\Delta_{local} Q(x) < \Delta_{local}
Q(y)$, $R -Q(x)>0$, $R - Q(y)>0$ and $\Delta_{local} Q(x)=Q(x)-Q(y)$. We have:
\begin{equation*}
\begin{split}
\frac{\epsilon |R - Q(y)|}{\Delta_{local} Q(y)} - \frac{\epsilon |R - Q(x)|}{\Delta_{local} Q(x)}
& = \frac{\epsilon (R - Q(y))}{\Delta_{local} Q(y)} - \frac{\epsilon (R - Q(x))}{\Delta_{local} Q(x)}\\
& > \frac{\epsilon (R - Q(y))}{\Delta_{local} Q(x)} - \frac{\epsilon (R - Q(x))}{\Delta_{local} Q(x)} \\
& = \frac{\epsilon((R - Q(y)) - (R - Q(x)))}{\Delta_{local} Q(x)}\\
% \end{split}
% \end{equation*}
% According to the triangle inequality,
% \begin{equation*}
% \begin{split}
% \frac{\epsilon|Q(y) - Q(x)|}{\Delta_{local} Q(x)} 
% & > \frac{\epsilon(|R - Q(y)| - |R - Q(x)|)}{\Delta_{local} Q(x)} 
& = \frac{\epsilon(Q(y) - Q(x))}{\Delta_{local} Q(x)} \\
% \epsilon & >\frac{\epsilon(|R - Q(y)| - |R - Q(x)|)}{\Delta_{local} Q(x)} > 
& = \epsilon.
\end{split}
\end{equation*}

i.e.
\begin{equation*}
privacy\ loss = \frac{P[Q^{*}(x) = R]}{P[Q^{*}(y) = R]} > exp(\epsilon),
\end{equation*}
% $privacy\ loss = \frac{P[Q^{*}(x) = R]}{P[Q^{*}(y) = R]} < exp(\epsilon)$ is untenable.

% \textcolor{red}{when the $\Delta_{local} Q(x) < \Delta_{local} Q(y) $, we have:
% \begin{equation*}
% \begin{split}
% \frac{\epsilon |R - Q(y)|}{\Delta_{local} Q(y)} - \frac{\epsilon |R - Q(x)|}{\Delta_{local} Q(x)}
% & < \frac{\epsilon |R - Q(y)|}{\Delta_{local} Q(x)} - \frac{\epsilon |R - Q(x)|}{\Delta_{local} Q(x)} \\
% & = \frac{\epsilon(|R - Q(x)| - |R - Q(y)|)}{\Delta_{local} Q(x)} \\
% & < \frac{\epsilon|Q(y) - Q(x)|}{\Delta_{local} Q(x)} \\
% & < \epsilon,
% \end{split}
% \end{equation*}}
% i.e.
% \begin{equation*}
% privacy\ loss = \frac{P[Q^{*}(x) = R]}{P[Q^{*}(y) = R]} < exp(\epsilon)
% \end{equation*}

and vice versa.

% \section{Exponential Mechanism under Local Sensitivity}
% The exponential mechanism scaled to local sensitivity is:

% Privacy loss analysis:



In conclusion, the privacy loss of Laplace noise under local sensitivity is not necessary smaller or equal to $e^{\epsilon}$, i.e. non-differentially private.

\end{document}