%%%%%%%% ICML 2018 EXAMPLE LATEX SUBMISSION FILE %%%%%%%%%%%%%%%%%

\documentclass{article}
% \usepackage[utf8]{inputenc}
% \usepackage[english]{babel}
% Recommended, but optional, packages for figures and better typesetting:
\usepackage{microtype}
\usepackage{graphicx}
\usepackage{amsmath}
\usepackage{amsthm}
\usepackage{mdframed}
\usepackage{subfigure}
\usepackage{booktabs} % for professional tables
% hyperref makes hyperlinks in the resulting PDF.
% If your build breaks (sometimes temporarily if a hyperlink spans a page)
% please comment out the following usepackage line and replace
% \usepackage{icml2018} with \usepackage[nohyperref]{icml2018} above.
\usepackage{hyperref}
% \renewcommand\qedsymbol{$\blacksquare$}
% \theoremstyle{plain}
%   \newtheorem{thm}{Theorem}
% \theoremstyle{definition}
%   \newtheorem*{proof1}{Proof of Theorem \ref{Fermat}}
% Attempt to make hyperref and algorithmic work together better:
\newcommand{\theHalgorithm}{\arabic{algorithm}}
% \newenvironment{proofof}[1]{\par
%   \pushQED{\qed}%
%   \normalfont \topsep6\p@\@plus6\p@\relax
%   \trivlist
%   \item[\hskip\labelsep
%         \bfseries
%     Proof of #1\@addpunct{.}]\ignorespaces
% }{%
%   \popQED\endtrivlist\@endpefalse
% }
% Use the following line for the initial blind version submitted for review:
\usepackage{icml2018}

% If accepted, instead use the following line for the camera-ready submission:
%\usepackage[accepted]{icml2018}

% The \icmltitle you define below is probably too long as a header.
% Therefore, a short form for the running title is supplied here:
\icmltitlerunning{Submission and Formatting Instructions for ICML 2018}

\begin{document}

\twocolumn[
\icmltitle{Differentially Private Bayesian Inference}

% It is OKAY to include author information, even for blind
% submissions: the style file will automatically remove it for you
% unless you've provided the [accepted] option to the icml2018
% package.

% List of affiliations: The first argument should be a (short)
% identifier you will use later to specify author affiliations
% Academic affiliations should list Department, University, City, Region, Country
% Industry affiliations should list Company, City, Region, Country

% You can specify symbols, otherwise they are numbered in order.
% Ideally, you should not use this facility. Affiliations will be numbered
% in order of appearance and this is the preferred way.
\icmlsetsymbol{equal}{*}

\begin{icmlauthorlist}
\icmlauthor{Aeiau Zzzz}{equal,to}
\icmlauthor{Bauiu C.~Yyyy}{equal,to,goo}
\icmlauthor{Cieua Vvvvv}{goo}
\icmlauthor{Iaesut Saoeu}{ed}
\icmlauthor{Fiuea Rrrr}{to}
\icmlauthor{Tateu H.~Yasehe}{ed,to,goo}
\icmlauthor{Aaoeu Iasoh}{goo}
\icmlauthor{Buiui Eueu}{ed}
\icmlauthor{Aeuia Zzzz}{ed}
\icmlauthor{Bieea C.~Yyyy}{to,goo}
\icmlauthor{Teoau Xxxx}{ed}
\icmlauthor{Eee Pppp}{ed}
\end{icmlauthorlist}

\icmlaffiliation{to}{Department of Computation, University of Torontoland, Torontoland, Canada}
\icmlaffiliation{goo}{Googol ShallowMind, New London, Michigan, USA}
\icmlaffiliation{ed}{School of Computation, University of Edenborrow, Edenborrow, United Kingdom}

\icmlcorrespondingauthor{Cieua Vvvvv}{c.vvvvv@googol.com}
\icmlcorrespondingauthor{Eee Pppp}{ep@eden.co.uk}

% You may provide any keywords that you
% find helpful for describing your paper; these are used to populate
% the "keywords" metadata in the PDF but will not be shown in the document
\icmlkeywords{Machine Learning, ICML}

\vskip 0.3in
]

% this must go after the closing bracket ] following \twocolumn[ ...

% This command actually creates the footnote in the first column
% listing the affiliations and the copyright notice.
% The command takes one argument, which is text to display at the start of the footnote.
% The \icmlEqualContribution command is standard text for equal contribution.
% Remove it (just {}) if you do not need this facility.

%\printAffiliationsAndNotice{}  % leave blank if no need to mention equal contribution
\printAffiliationsAndNotice{\icmlEqualContribution} % otherwise use the standard text.

\begin{abstract}
\end{abstract}

\section{Setting up}
\label{sec_setup}
The Bayesian inference process is denoted as $\mathsf{BayesInfer}(x,prior)$ taking an observed data set $x \in \mathcal{X}^n$ and a prior distribution as input, outputting a posterior distribution $posterior$. For conciseness, when prior is given, we use $\mathsf{BayesInfer}(x)$.

For now, we already have a prior distribution $prior$, an observed data set $x$.

\subsection{Exponential Mechanism with Global Sensitivity}
\label{subsec_emgs}

\subsubsection{Mechanism Set up}
In exponential mechanism, candidate set $R$ can be obtained by enumerating $y \in \mathcal{X}^n$, i.e.
\begin{equation*}
R = \{\mathsf{BayesInfer}(y)\ |\ y \in \mathcal{X}^n\}.
\end{equation*}

Hellinger distance $\mathsf{Hlg}$ is used here to score these candidates. The utility function:
\begin{equation}
\label{equ_utility}
u(x,r) = -\mathsf{Hlg}(\mathsf{BayesInfer}(x), r); r \in R.
\end{equation}

Exponential mechanism with global sensitivity selects and outputs a candidate $r \in R$ with probability proportional to $exp(\frac{\epsilon u(x,r)}{2 \Delta_{g}u})$:
\begin{equation*}
P[r] = \frac
{exp(\frac{\epsilon u(x,r)}{2 \Delta_{g}u})}
{\Sigma_{r' \in R}\ exp(\frac{\epsilon u(x,r')}{2 \Delta_{g}u})},
\end{equation*}
where global sensitivity is calculated by:
\begin{multline*}
\Delta_{g}u = 
\mathsf{Hlg}(\mathsf{BayesInfer}(x'), r) - \mathsf{Hlg}(\mathsf{BayesInfer}(y'), r)|\\
\max_{\{|x',y'| \leq 1;x',y'\in \mathcal{X}^n\}}\max_{\{r\in R\}}.
\end{multline*}

% We use probability proportional to $exp(\frac{\epsilon u(x,r)}{\Delta_{g}u})$ rather than $exp(\frac{\epsilon u(x,r)}{2 \Delta_{g}u})$. Since $\mathsf{Hlg}$ is monotonic, i.e. our utility function is monotonic. 

\subsubsection{Security Analysis}
It can be proved that exponential mechanism with global sensitivity is $\epsilon$-differentially private. We denote the $\mathsf{BayesInfer}$ with privacy mechanism as $\mathsf{PrivInfer}$. For adjacent data set $||x,y||_1 = 1$:
\begin{equation*}
\begin{split}
& \frac{P[\mathsf{PrivInfer}(x,u,R) = r]}{P[\mathsf{PrivInfer}(y,u,R) = r]} \\
& =\frac
{\frac
{exp(\frac{\epsilon u(x,r)}{2 \Delta_{g}u})}
{\Sigma_{r' \in R}\ exp(\frac{\epsilon u(x,r')}{2 \Delta_{g}u})}}
{\frac
{exp(\frac{\epsilon u(y,r)}{2 \Delta_{g}u})}
{\Sigma_{r' \in R}\ exp(\frac{\epsilon u(y,r')}{2 \Delta_{g}u})}} \\
& = \left(\frac
{exp(\frac{\epsilon u(x,r)}{2 \Delta_{g}u})}
{exp(\frac{\epsilon u(y,r)}{2 \Delta_{g}u})}
\right)
\cdot
\left(\frac
{\sum\limits_{r' \in R}\ exp(\frac{\epsilon u(y,r')}{2 \Delta_{g}u})}
{\sum\limits_{r' \in R}\ exp(\frac{\epsilon u(x,r')}{2 \Delta_{g}u})}
\right)\\
& = exp\left(\frac
{\epsilon (u(x,r) - u(y,r))}
{2 \Delta_{g}u}
\right) \\
& \cdot
\left(\frac
{\sum\limits_{r' \in R}\ exp(\frac{\epsilon u(y,r')}{2 \Delta_{g}u})}
{\sum\limits_{r' \in R}\ exp(\frac{\epsilon u(x,r')}{2 \Delta_{g}u})}
\right)\\
& \leq
exp(\frac{\epsilon}{2}) \cdot exp(\frac{\epsilon}{2}) \cdot
\left(\frac
{\sum\limits_{r' \in R}\ exp(\frac{\epsilon u(x,r')}{2 \Delta_{g}u})}
{\sum\limits_{r' \in R}\ exp(\frac{\epsilon u(x,r')}{2 \Delta_{g}u})}
\right)\\
& = exp(\epsilon).
\end{split}
\end{equation*}

Then, $\frac{P[\mathsf{PrivInfer}(x,u,R) = r]}{P[\mathsf{PrivInfer}(y,u,R) = r]} \geq exp(-\epsilon)$ can be obtained by symmetry.


\subsection{Exponential Mechanism with Local Sensitivity}
\label{subsec_emls}
\subsubsection{Mechanism Set up}
Exponential mechanism with local sensitivity share the same candidate set and utility function as it with global sensitivity. This outputs a candidate $r \in R$ with probability proportional to $exp(\frac{\epsilon u(x,r)}{2 \Delta_{l}u})$:
\begin{equation*}
P[r] = \frac
{exp(\frac{\epsilon u(x,r)}{2 \Delta_{l}u})}
{\Sigma_{r' \in R}\ exp(\frac{\epsilon u(x,r')}{2 \Delta_{l}u})},
\end{equation*}

where local sensitivity is calculated by:

\begin{multline*}
\Delta_{g}u(x) = 
\mathsf{Hlg}(\mathsf{BayesInfer}(x), r) - \mathsf{Hlg}(\mathsf{BayesInfer}(y'), r)|\\
\max_{\{|x,y'| \leq 1;y'\in \mathcal{X}^n\}}\max_{\{r\in R\}}.
\end{multline*}

\subsubsection{Security Analysis}
We will then prove that exponential mechanism with local sensitivity is non-differentialy private.
\begin{equation*}
\begin{split}
& \frac{P[\mathsf{PrivInfer}(x,u,R) = r]}{P[\mathsf{PrivInfer}(y,u,R) = r]} \\
& = exp\left(
\frac{\epsilon u(x,r)}{2 \Delta_{l}u(x)} - 
\frac{\epsilon u(y,r)}{2 \Delta_{l}u(y)}
\right)  \cdot
\left(\frac
{\sum\limits_{r' \in R}\ exp(\frac{\epsilon u(y,r')}{2 \Delta_{l}u(y)})}
{\sum\limits_{r' \in R}\ exp(\frac{\epsilon u(x,r')}{2 \Delta_{l}u(x)})}
\right) \\
& = \frac
{\sum\limits_{r' \in R}\ exp(\frac{\epsilon u(x,r)}{2 \Delta_{l}u(x)} 
+ \frac{\epsilon u(y,r')}{2 \Delta_{l}u(y)})}
{\sum\limits_{r' \in R}\ exp(\frac{\epsilon u(y,r)}{2 \Delta_{l}u(y)} 
+ \frac{\epsilon u(x,r')}{2 \Delta_{l}u(x)})}.
\end{split}
\end{equation*}

Without loss of generality, we consider the case that $\Delta_{l}u(y) < \Delta_{l}u(x)$, $r = arg(\max\limits_{r' \in R}\{u(x,r')\}) = arg(\min\limits_{r' \in R}\{u(y,r')\})$ and $\Delta_{l}u(y) = u(x,r) - u(y,r)$. We have:
\begin{equation*}
\begin{split}
& \frac
{\sum\limits_{r' \in R}\ exp(\frac{\epsilon u(x,r)}{2 \Delta_{l}u(x)} 
+ \frac{\epsilon u(y,r')}{2 \Delta_{l}u(y)})}
{\sum\limits_{r' \in R}\ exp(\frac{\epsilon u(y,r)}{2 \Delta_{l}u(y)} 
+ \frac{\epsilon u(x,r')}{2 \Delta_{l}u(x)})}\\
& > \frac
{\sum\limits_{r' \in R}\ exp(\frac{\epsilon (u(x,r) + u(y,r'))}{2 \Delta_{l}u(x)})}
{\sum\limits_{r' \in R}\ exp(\frac{\epsilon (u(y,r) + u(x,r'))}{2 \Delta_{l}u(y)})}\\
& > \frac
{|R|\ exp(\frac{\epsilon (u(x,r) + u(y,r))}{2 \Delta_{l}u(x)})}
{|R|\ exp(\frac{\epsilon (u(y,r) + u(x,r))}{2 \Delta_{l}u(y)})}\\
& = exp(\frac{\epsilon}{2} (\frac{u(x,r) + u(y,r)}{\Delta_{l}u(x)} - \frac{u(x,r) + u(y,r)}{\Delta_{l}u(y)})).
\end{split}
\end{equation*}

From Eq. \ref{equ_utility}, $\{u(x,r') \leq 0|r' \in R\}$ and $\{u(y,r') \leq 0|r' \in R\}$, we can infer that $r = arg(\max\limits_{r \in R}\{u(x,r')\}) = \mathsf{BayesInfer}(x)$ and $u(x,r) = 0$.From $\Delta_{l}u(y) = u(x,r) - u(y,r)$, we can also infer that $\Delta_{l}u(y) = - u(y,r)$. Then, the following relationship between $u(x,r)$, $u(y,r)$, $\Delta_{l}u(x)$ and $\Delta_{l}u(y)$:
\begin{equation*}
\begin{split}
- \Delta_{l}u(x) & < \Delta_{l}u(y) \\
\Delta_{l}u(x) - \Delta_{l}u(y) & < 2 \Delta_{l}u(x) \\
- \Delta_{l}u(y) (\Delta_{l}u(y) - \Delta_{l}u(x)) 
& < 2 \Delta_{l}u(x) \Delta_{l}u(y) \\
u(y,r) (\Delta_{l}u(y) - \Delta_{l}u(x)) 
& < 2 \Delta_{l}u(x) \Delta_{l}u(y) \\
\frac{u(x,r) + u(y,r)}{\Delta_{l}u(x)} - \frac{u(x,r) + u(y,r)}{\Delta_{l}u(y)} & > 2.
\end{split}
\end{equation*}

% \begin{equation*}
% \begin{split}
% \frac{\Delta_{l}u(x)}{\Delta_{l}u(y)} & > 0\\
% & > \frac{\Delta_{l}u(x)}{2u(y,r)}\\
% & = \frac{\Delta_{l}u(x) + 0}{2u(y,r)} \\
% & = \frac{\Delta_{l}u(x) + 2 u(x,r)}{2u(y,r)}
% \end{split}
% \end{equation*}
holds.
 
Then we can have:
\begin{equation*}
\begin{split}
& exp(\frac{\epsilon}{2} (\frac{u(x,r) + u(y,r)}{\Delta_{l}u(y)} - \frac{u(x,r) + u(y,r)}{\Delta_{l}u(x)}))\\
& > exp(\frac{\epsilon}{2} * 2) \\
& = exp(\epsilon),
\end{split}
\end{equation*}
i.e.
\begin{equation*}
\begin{split}
\frac{P[\mathsf{PrivInfer}(x,u,R) = r]}{P[\mathsf{PrivInfer}(y,u,R) = r]}
& > exp(\epsilon).
\end{split}
\end{equation*}

Since there are cases where exponential mechanism with local sensitivity's privacy loss is greater than $e^{\epsilon}$. we can say it is non-differentially private.
% \begin{equation*}
% \begin{split}
% & \epsilon = 0.1\\
% data size = 2\\
% the prior is Beta(2,2)\\
% data set x = [1,0,]\\
% data set y = [0,0]\\
% -0.313380201461\\
% -0.0\\
% -0.313380201461\\
% 3 3\\
% 0.313380201461\\
% 0.322043464396\\
% 0.355913071207\\
% 0.322043464396\\
% -0.0\\
% -0.313380201461\\
% -0.57735026919\\
% 2 4\\
% 0.375460728684\\
% 0.360050126956\\
% 0.331218326999\\
% 0.308731546045
% \end{split}
% \end{equation*}

% Without loss of generality, we consider the case that $\Delta_{l}u(y) > \Delta_{l}u(x)$ and $\Delta_{l}u(x) = u(x,r) - u(y,r)$. We have:
% \begin{equation*}
% \begin{split}
% exp\left(
% \frac{\epsilon u(x,r)}{2 \Delta_{l}u(x)}
% - \frac{\epsilon u(y,r)}{2 \Delta_{l}u(y)}
% \right)
% & > exp\left(
% \frac{\epsilon u(x,r)}{2 \Delta_{l}u(x)}
% - \frac{\epsilon u(y,r)}{2 \Delta_{l}u(x)}
% \right)\\
% & = exp\left(
% \frac{\epsilon u(x,r) - \epsilon u(y,r)}{2 \Delta_{l}u(x)}
% \right) \\
% & = exp(\frac{\epsilon}{2});
% \end{split}
% \end{equation*}

% \textcolor{red}{For the second part:
% , we have:
% \begin{equation*}
% \begin{split}
% \frac{\epsilon u(y,r')}{2 \Delta_{l}(y)} - \frac{\epsilon u(x,r')}{2 \Delta_{l}(x)} 
% & < \frac{\epsilon u(y,r')}{2 \Delta_{l}(y)} - \frac{\epsilon u(x,r')}{2 \Delta_{l}(y)}\\
% & \leq \frac{\epsilon}{2}
% \end{split}
% \end{equation*}
% I cannot replace $exp(\frac{\epsilon u(y,r')}{2 \Delta_{l}(y)})$ in $\left(\frac
% {\sum\limits_{r' \in R}\ exp(\frac{\epsilon u(y,r')}{2 \Delta_{l}u(y)})}
% {\sum\limits_{r' \in R}\ exp(\frac{\epsilon u(x,r')}{2 \Delta_{l}u(x)})}
% \right)$ with $ > exp(\frac{\epsilon}{2}) \cdot exp(\frac{\epsilon u(x,r')}{2 \Delta_{l}(x)})$.
% So, I cannot find the relationship between the whole part and $exp(\epsilon)$.}

% For the second part:
% \begin{equation*}
% \begin{split}
% \left(\frac
% {\sum\limits_{r' \in R}\ exp(\frac{\epsilon u(y,r')}{2 \Delta_{l}u(y)})}
% {\sum\limits_{r' \in R}\ exp(\frac{\epsilon u(x,r')}{2 \Delta_{l}u(x)})}
% \right)
% & < \left(\frac
% {\sum\limits_{r' \in R}\ exp(\frac{\epsilon u(y,r')}{2 \Delta_{l}u(x)})}
% {\sum\limits_{r' \in R}\ exp(\frac{\epsilon u(x,r')}{2 \Delta_{l}u(x)})}
% \right)\\
% & < \left(\frac
% {exp(\frac{\epsilon}{2})
% \cdot
% \sum\limits_{r' \in R}\ exp(\frac{\epsilon u(x,r')}{2 \Delta_{l}u(x)})}
% {\sum\limits_{r' \in R}\ exp(\frac{\epsilon u(x,r')}{2 \Delta_{l}u(x)})}
% \right)\\
% & < exp(\frac{\epsilon}{2})
% \end{split}
% \end{equation*}

% For the second part(a tighter bound):
% \begin{equation*}
% \begin{split}
% & \left(\frac
% {\sum\limits_{r' \in R}\ exp(\frac{\epsilon u(y,r')}{2 \Delta_{l}u(y)})}
% {\sum\limits_{r' \in R}\ exp(\frac{\epsilon u(x,r')}{2 \Delta_{l}u(x)})}
% \right) \\
% & = \left(\frac
% {\sum\limits_{r' \in R}\ exp(\frac{\epsilon u(y,r')}{2 \Delta_{l}u(x)})}
% {exp(\frac{\Delta _{l}(y) - \Delta _{l}(x)}{\Delta _{l}(x)  \Delta _{l}(y)})
% \cdot
% \sum\limits_{r' \in R}\ exp(\frac{\epsilon u(x,r')}{2 \Delta_{l}u(x)})}
% \right)\\
% & < \left(\frac
% {exp(\frac{\epsilon}{2})
% \cdot
% \sum\limits_{r' \in R}\ exp(\frac{\epsilon u(x,r')}{2 \Delta_{l}u(x)})}
% {exp(\frac{\Delta _{l}(y) - \Delta _{l}(x)}{\Delta _{l}(x)  \Delta _{l}(y)})
% \cdot
% \sum\limits_{r' \in R}\ exp(\frac{\epsilon u(x,r')}{2 \Delta_{l}u(x)})}
% \right)\\
% & < \left(\frac
% {exp(\frac{\epsilon}{2})}
% {exp(\frac{\Delta _{l}(y) - \Delta _{l}(x)}{\Delta _{l}(x)  \Delta _{l}(y)})}
% \right)\\
% & < exp(\frac{\epsilon}{2} - \frac{\Delta _{l}(y) - \Delta _{l}(x)}{\Delta _{l}(x)  \Delta _{l}(y)})
% \end{split}
% \end{equation*}


\subsection{Exponential Mechanism of Varying Sensitivity}
\label{subsec_emvs}
\subsubsection{Mechanism Setting up}

\subsubsection{Security Analysis}

\subsection{Exponential Mechanism of Smooth Sensitivity}
\label{subsec_emss}
\subsubsection{Mechanism Setting up}

\subsubsection{Security Analysis}


\section{Privacy Fix}

\subsection{Propositions}
Assume we have a prior distribution $beta(1,1)$, an observed data set $x \in \{0,1\}^n$, $n>0$. We use the $x+1$ and $x-1$ to denote:
\begin{small}
\begin{equation*}
\begin{split}
&\mathsf{if\ BayesInfer}(x) = beta(a_1 + 1,b_1 + 1)\\
&\mathsf{then}\mathsf{\ BayesInfer}(x + 1) = beta((a_1 + 1) +1, (b_1 - 1)+1)\\
&\quad \quad \mathsf{\ BayesInfer}(x - 1) = beta((a_1 - 1) +1, (b_1 + 1)+1),\\
\end{split}
\end{equation*}
\end{small}

$x_0$ to denote:
\begin{small}
\begin{equation*}
\begin{split}
&\mathsf{if}\ n\ is\ even\\
&\mathsf{then}\ \mathsf{BayesInfer}(x_0) = beta(\frac{n}{2} + 1, \frac{n}{2} + 1)\\
&\mathsf{else}\ \mathsf{BayesInfer}(x_0) = \{beta(\frac{n+1}{2} + 1, \frac{n-1}{2} + 1),\\
&\quad \quad \quad \quad \quad \quad \quad \quad \quad \quad beta(\frac{n - 1}{2} + 1, \frac{n + 1}{2} + 1)\}
\end{split}
\end{equation*}
\end{small}

Then, we have the following three statements, and proofs of the statements.
\renewcommand{\labelenumi}{\Roman{enumi}}
\begin{enumerate}
\item $\mathsf{Hlg} (\mathsf{BayesInfer}(x), \mathsf{BayesInfer}(x + 1)) < \mathsf{Hlg} (\mathsf{BayesInfer}(x + 1), \mathsf{BayesInfer}(x + 2))\ \forall x \geq x_0$;

or $\mathsf{Hlg} (\mathsf{BayesInfer}(x), \mathsf{BayesInfer}(x + 1)) > \mathsf{Hlg} (\mathsf{BayesInfer}(x + 1), \mathsf{BayesInfer}(x + 2))\forall x \leq x_0$.

\item $\Delta_{l}u(x) = \mathsf{Hlg} (\mathsf{BayesInfer}(x),\mathsf{BayesInfer}(x + 1)),\forall x \geq x_0$;

$\Delta_{l}u(x) = \mathsf{Hlg} (\mathsf{BayesInfer}(x),\mathsf{BayesInfer}(x - 1)),\forall x \leq x_0$.

\item $\forall x \neq x_0: \Delta_{l}u(x) > \Delta_{l}u(x_0)$.
\end{enumerate}

\subsection{proof}

\subsubsection{Statement I}
We use the MI (Mathematical Induction) method to prove the first statement.

\begin{proof}
Since the Hellinger distance is symmetric, if we prove the $\mathsf{Hlg} (\mathsf{BayesInfer}(x), \mathsf{BayesInfer}(x + 1)) < \mathsf{Hlg} (\mathsf{BayesInfer}(x + 1), \mathsf{BayesInfer}(x + 2))\ \forall x \geq x_0$, the other part when $\forall x \leq x_0$ also holds.

1. if $x = x_0$, $\mathsf{Hlg} (\mathsf{BayesInfer}(x_0), \mathsf{BayesInfer}(x_0 + 1)) < \mathsf{Hlg} (\mathsf{BayesInfer}(x_0 + 1), \mathsf{BayesInfer}(x_0 + 2))$ holds:


% \twocolumn{
\begin{equation*}
\boxed{
\begin{split}
\mathsf{Hlg}(beta(\frac{n}{2} + 1, \frac{n}{2} + 1), beta(\frac{n}{2} + 1 + 1, \frac{n}{2} + 1 - 1))
& <  \mathsf{Hlg}(beta(\frac{n}{2} + 1 + 1, \frac{n}{2} + 1 - 1), beta(\frac{n}{2} + 1 + 2, \frac{n}{2} + 1 - 2))\\
\sqrt{1 - \frac{\mathsf{beta}(\frac{\frac{n}{2} + 1 + \frac{n}{2} + 1 + 1}{2},\frac{\frac{n}{2} + 1 + \frac{n}{2} + 1 - 1}{2})}
{\sqrt{\mathsf{beta}(\frac{n}{2} + 1, \frac{n}{2} + 1)\mathsf{beta}(\frac{n}{2} + 1 + 1, \frac{n}{2} + 1 - 1)}}}
& < \sqrt{1 - \frac{\mathsf{beta}(\frac{\frac{n}{2} + 1 + 1 + \frac{n}{2} + 1 + 2}{2}, \frac{\frac{n}{2} + 1 - 1 + \frac{n}{2} + 1 - 2}{2}, )}
{\sqrt{\mathsf{beta}(\frac{n}{2} + 1 + 1, \frac{n}{2} + 1 - 1)\mathsf{beta}(\frac{n}{2} + 1 + 2, \frac{n}{2} + 1 - 2)}}}\\
\sqrt{1 - \frac{\mathsf{beta}(\frac{n + 3}{2},\frac{n + 1}{2})}
{\sqrt{\mathsf{beta}(\frac{n}{2} + 1, \frac{n}{2} + 1)\mathsf{beta}(\frac{n}{2} + 2, \frac{n}{2})}}}
& < \sqrt{1 - \frac{\mathsf{beta}(\frac{n + 5}{2}, \frac{n - 1}{2})}
{\sqrt{\mathsf{beta}(\frac{n}{2} + 2, \frac{n}{2})\mathsf{beta}(\frac{n}{2} + 3, \frac{n}{2} - 1)}}}\\
\frac{\mathsf{beta}(\frac{n + 3}{2},\frac{n + 1}{2})}
{\sqrt{\mathsf{beta}(\frac{n}{2} + 1, \frac{n}{2} + 1)\mathsf{beta}(\frac{n}{2} + 2, \frac{n}{2})}}
& > \frac{\mathsf{beta}(\frac{n + 5}{2}, \frac{n - 1}{2})}
{\sqrt{\mathsf{beta}(\frac{n}{2} + 2, \frac{n}{2})\mathsf{beta}(\frac{n}{2} + 3, \frac{n}{2} - 1)}}\\
\frac{\mathsf{beta}(\frac{n + 3}{2},\frac{n - 1}{2}) \frac{\frac{n - 1}{2}}{\frac{n - 1}{2} + \frac{n + 3}{2}}}
{\sqrt{\mathsf{beta}(\frac{n}{2} + 1, \frac{n}{2} - 1) \frac{\frac{n}{2} - 1}{\frac{n}{2} - 1 + \frac{n}{2} + 1} \frac{\frac{n}{2}}{\frac{n}{2} + \frac{n}{2} + 1}}}
& > \frac{\mathsf{beta}(\frac{n + 3}{2}, \frac{n - 1}{2}) \frac{\frac{n + 3}{2}}{\frac{n + 3}{2} + \frac{n - 1}{2}} }
{\sqrt{\mathsf{beta}(\frac{n}{2} + 1, \frac{n}{2} - 1) \frac{\frac{n}{2} + 1}{\frac{n}{2} + 1 + \frac{n}{2} - 1} \frac{\frac{n}{2} + 2}{\frac{n}{2} + 2 + \frac{n}{2} - 1}}}\\
\frac{\frac{n - 1}{2}}{\sqrt{(\frac{n}{2} - 1)(\frac{n}{2})}}
& > \frac{\frac{n + 3}{2}}{\sqrt{(\frac{n}{2} + 1)(\frac{n}{2} + 2)}}\\
(n - 1) ^ 2 (n + 2)(n + 4)
& > (n + 3) ^ 2 n (n - 2)\\
n & > -1.
\end{split}
}
\end{equation*}

Since $n > 0$, it always holds.

2. if $x = x_0 + m$ holds, then also $x = x_0 + m + 1$ holds:

i.e $\mathsf{Hlg}(beta(\frac{n}{2} + 1 + m, \frac{n}{2} + 1 - m), beta(\frac{n}{2} + 1 + m + 1, \frac{n}{2} + 1 - m - 1)) 
< \mathsf{Hlg}(beta(\frac{n}{2} + 1 + m + 1, \frac{n}{2} + 1 - m - 1), beta(\frac{n}{2} + 1 + m + 2, \frac{n}{2} + 1 - m - 2))$
is what we know:

\begin{equation*}
\begin{split}
& \sqrt{1 - \frac{\mathsf{beta}(
\frac{\frac{n}{2} + 1 + m + \frac{n}{2} + 1 + m + 1}{2},
\frac{\frac{n}{2} + 1 - m + \frac{n}{2} + 1 - m - 1}{2})}
{\sqrt{\mathsf{beta}(\frac{n}{2} + 1 + m, \frac{n}{2} + 1 - m)
\mathsf{beta}(\frac{n}{2} + 2 + m, \frac{n}{2} - m)}}}\\
& < \sqrt{1 - \frac{\mathsf{beta}(,)}{\sqrt{\mathsf{beta}(,) \mathsf{beta}(,)}}}\\
& \frac{\mathsf{beta}(,)}{\sqrt{\mathsf{beta}(,) \mathsf{beta}(,)}}\\
& > \frac{\mathsf{beta}(,)}{\sqrt{\mathsf{beta}(,) \mathsf{beta}(,)}}
\end{split}
\end{equation*}


Now, we want to proof $\mathsf{Hlg}(beta(\frac{n}{2} + 1 + m + 1, \frac{n}{2} + 1 - m - 1), beta(\frac{n}{2} + 1 + m + 2, \frac{n}{2} + 1 - m - 2)) 
< \mathsf{Hlg}(beta(\frac{n}{2} + 1 + m + 2, \frac{n}{2} + 1 - m - 2), beta(\frac{n}{2} + 1 + m + 3, \frac{n}{2} + 1 - m - 3))$ by using what we know:
\end{proof}

\subsubsection{Statement II}

\subsubsection{Statement III}
\vspace*{10cm}

\section{Experimental Evaluations}
We got some results from these mechanisms.


% Acknowledgements should only appear in the accepted version.
% \section*{Acknowledgements}
% In the unusual situation where you want a paper to appear in the
% references without citing it in the main text, use \nocite
% \nocite{langley00}

\bibliographystyle{icml2018}
\bibliography{example_paper}





\end{document}


% This document was modified from the file originally made available by
% Pat Langley and Andrea Danyluk for ICML-2K. This version was created
% by Iain Murray in 2018. It was modified from a version from Dan Roy in
% 2017, which was based on a version from Lise Getoor and Tobias
% Scheffer, which was slightly modified from the 2010 version by
% Thorsten Joachims & Johannes Fuernkranz, slightly modified from the
% 2009 version by Kiri Wagstaff and Sam Roweis's 2008 version, which is
% slightly modified from Prasad Tadepalli's 2007 version which is a
% lightly changed version of the previous year's version by Andrew
% Moore, which was in turn edited from those of Kristian Kersting and
% Codrina Lauth. Alex Smola contributed to the algorithmic style files.
