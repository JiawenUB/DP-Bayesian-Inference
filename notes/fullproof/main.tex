\documentclass[sigconf]{acmart}

\fancyhf{} % Remove fancy page headers 
% \fancyhead[C]{Anonymous submission \#9999 to ACM CCS 2017} % TODO: replace 9999 with your paper number
\fancyfoot[C]{\thepage}

\setcopyright{none} % No copyright notice required for submissions
% \acmConference[Anonymous Submission to ACM CCS 2017]{ACM Conference on Computer and Communications Security}{Due 19 May 2017}{Dallas, Texas}
% \acmYear{2017}

\settopmatter{printacmref=false, printccs=true, printfolios=true} % We want page numbers on submissions

%%\ccsPaper{9999} % TODO: replace with your paper number once obtained

%packages
\usepackage{natbib}
\usepackage{amsmath}
\usepackage{amsthm}
\usepackage{mathtools}
\usepackage{mdframed}
\usepackage{subfigure}
\usepackage{booktabs}
% \usepackage{hyperref}
\usepackage{subfigure}
\usepackage{siunitx} % Provides the \SI{}{} and \si{} command for typesetting SI units
\usepackage{graphicx} % Required for the inclusion of images
% \usepackage{natbib} % Required to change bibliography style to APA
\usepackage{datetime}
\usepackage{lscape}
\usepackage{algorithm}
\usepackage{algorithmic}
\usepackage{xspace}
\usepackage[english]{babel} % English language/hyphenation
\usepackage{proof}
\usepackage{booktabs} % Top and bottom rules for tables
\usepackage[colorlinks, allcolors = blue,]{hyperref}
\usepackage{accents}
\usepackage{amsfonts}
\usepackage{stmaryrd}
\usepackage{amsmath,amsthm,amssymb,latexsym} 
\usepackage{microtype}
\usepackage{graphicx}
\usepackage{subfigure}
\usepackage{booktabs} % for professional tables
\usepackage{hyperref}
\usepackage{icml2019}
\usepackage{lipsum}

\usepackage{authblk}


%new commands
\newcommand{\theHalgorithm}{\arabic{algorithm}}
\newtheorem{definition}{Definition}
\usepackage{cancel}
\usepackage[normalem]{ulem}
\newcommand{\dataobs}{\textbf{x}}
\newcommand{\adj}[2]{\textbf{adj}(#1,#2)}
\newcommand{\candidateset}{\mathcal{R}_{\textup{post}}}
\newcommand{\bprior}{\boldsymbol{\beta}_{\textup{prior}}}
\newcommand{\bysinfer}{\mathsf{Infer}}
\newcommand{\betad}{\mathsf{Beta}}
\newcommand{\betaf}{\textup{B}}
\newcommand{\mbetaf}{\boldsymbol{\textup{B}}}
\newcommand{\vtheta}{\boldsymbol{\theta}}
\newcommand{\valpha}{\boldsymbol{\alpha}}
\newcommand{\vbeta}{\boldsymbol{\beta}}
\newcommand{\lapmech}{\mathsf{LSDim}}
\newcommand{\ilapmech}{\mathsf{LSHist}}
\newcommand{\binomial}[2]{\mathsf{Bin}(#1, #2)}
\newcommand{\multinomial}[2]{\mathsf{Mult}(#1, #2)}
\newcommand{\expmech}{\mathsf{EHD}}
\newcommand{\hexpmech}{\mathsf{EHDS}}
\newcommand{\lexpmech}{\mathsf{EHDL}}
\newcommand{\hexpmechd}{\mathsf{expMech}^{D}_{\hellinger}}
\newcommand{\privinfer}{\mathsf{PrivInfer}}
\newcommand{\hlg}{\mathsf{H}}
\newcommand{\dirichlet}[1]{\mathsf{Dir}(#1)}
\newcommand{\alphas}{\boldsymbol{\alpha}}
\newcommand{\xis}{\boldsymbol{\xi}}
\newcommand{\iverson}[1]{[#1]}
\newcommand{\datauni}{\mathcal{X}}
\newcommand{\hellinger}{\mathcal{H}}
\newcommand{\ux}[1]{u(\textbf{x}, {#1})}
\newcommand{\uxadj}[1]{u(\textbf{x}', {#1})}
\newcommand{\cardinality}[2]{\mathcal{C}^{#1}_{#2}}
\newcommand{\range}{\mathcal{O}}
\newcommand{\nomalizer}[1]{\sum\limits_{r'\in \mathcal{R}_{\textup{post}}} \exp \big(\frac{-\epsilon\cdot \mathcal{H} (\mathsf{BI}(#1),r')}{4 \cdot S(#1)}\big)}

\newcommand{\unomalizer}[1]{\sum\limits_{r'\in \mathcal{R}_{\textup{post}}} \exp \big(\frac{-\epsilon\cdot u(#1, r')}{4 \cdot S(#1)}\big)}


\newcommand{\hexpmechPr}[2]{\underset{z \thicksim \hexpmech(#1)}{\Pr}\left[ #2 \right]}
\newcommand{\lapmechPr}[2]{\underset{z \thicksim \lapmech(#1)}{\Pr}\left[ #2 \right]}

\newcommand{\ilapmechPr}[2]{\underset{
{z \thicksim \ilapmech(#1)}
}{\Pr}\left[ #2 \right]}

\newtheorem{thm}{Theorem}[section]

\newtheorem{lem}{Lemma}[section]

\newtheorem{assert}{Assertion}[lem]
\newcommand{\lap}[2]{\mathsf{Lap}(#1, #2)}
\newcommand{\todo}[1]{{\footnotesize \color{red}\textbf{[[ #1 ]]}}}
\usepackage{accents}

\begin{document}
\title{Full proof on the differential privacy property of $\hexpmech$}


\begin{abstract}
\end{abstract}


\keywords{Differential privacy, Bayesian inference, Hellinger distance} % TODO: replace with your keywords

\maketitle


\section{Preliminaries}
\label{sec_background}

\noindent \textbf{Bayesian Inference.}

Given a prior belief $\Pr(\theta)$ on some parameter $\theta$,
and an observation $\dataobs$, the posterior distribution on $\theta$ given $\dataobs$ is computed as:
\[
  \Pr(\theta | \dataobs) = \frac{\Pr(\dataobs | \theta) \cdot \Pr(\theta)}{\Pr(\dataobs)}
\]
where the expression $\Pr(\dataobs | \theta)$ denotes the
\emph{likelihood} of observing $\dataobs$ under a value of
$\theta$. Since we consider $\dataobs$ to be fixed, the likelihood is
a function of $\theta$.
For the same reason $\Pr(\dataobs)$ is a constant independent of $\theta$.
Usually in statistics the prior distribution $\Pr(\theta)$ is chosen so that it represents
the initial belief on $\theta$, that is, when no data has been observed. In practice though,
prior distributions and likelihood functions are usually chosen so that the posterior
belongs to the same \emph{family} of distributions. In this case we say that the prior
is conjugate to the likelihood function. Use of a conjugate prior
simplifies calculations and allows for inference to be performed in a
recursive fashion over the data.


\noindent \textbf{Beta-binomial System.}

In this work we will consider a specific instance of Bayesian inference and one of its generalizations.
specifically, a Beta-binomial mode. We will consider the situation the underlying data is binomial distribution ($\sim binomial(\theta)$), where $\theta$ represents
the parameter --informally called \emph{bias}-- of a Bernoulli
distributed random variable. The
prior distribution over $\theta\in [0,1]$ is going to be a beta
distribution, $\betad(\alpha, \beta)$, with parameters
$\alpha,\beta\in\mathbb{R}^{+}$, and with p.d.f:
\[
  \Pr(\theta)\equiv \frac{\theta^{\alpha} (1- \theta)^{\beta}}{\betaf(\alpha,\beta)}
\]
where $\betaf(\cdot,\cdot)$ is the beta function.
The data $\dataobs$ will be a sequence of $n\in\mathbb{N}$ binary values, that is $\dataobs= (x_1,\dots x_n), x_i\in\{0,1\}$, and the likelihood function is:
\[
  \Pr(\dataobs | \theta)\equiv \theta^{\Delta \alpha}(1-\theta)^{n - \Delta \alpha}
\]
where $\Delta \alpha = \displaystyle\sum_{i=1}^{n}x_i$.
From this it can easily be derived that the posterior distribution is:
\[
  \Pr(\theta|\dataobs)=\betad(\alpha + \Delta \alpha,\beta + n - \Delta \alpha)
\]


\noindent \textbf{Dirichlet-multinomial Systems.}

The beta-binomial model can be immediately generalized to Dirichlet-multinomial, with underlying data multinomially distributed. The \emph{bias} is represented by parameter $\vtheta$, the vector of parameters of a categorically distributed random variable. The prior distribution over $\vtheta\in [0,1]^{k}$
is given by a Dirichelet distribution, $\dirichlet(\valpha)$, for $k\in\mathbb{N}$,
and $\valpha\in(\mathbb{R}^{+})^{k}$, with p.d.f:
\[
  \Pr(\vtheta)\equiv\frac{1}{\mbetaf(\valpha)}\cdot \displaystyle\prod_{i=1}^{k}{\theta_i^{\alpha_i-1}}
\]
where $\mbetaf(\cdot)$ is the generalized beta function.
The data $\dataobs$ will be a sequence of $n\in\mathbb{N}$ values
coming from a universe $\datauni$, such that $\mid\datauni \mid=k$.
The likelihood function will be:
\[
  \Pr(\dataobs|\vtheta)\equiv \displaystyle\prod_{a_i\in\datauni}\theta_{i}^{\Delta \alpha_i},
\]
with $\Delta \alpha_i=\displaystyle\sum_{j=1}^{n}\iverson{x_j=a_i}$, where $\iverson{\cdot}$ represents Iverson bracket notation.
Denoting by $\Delta\valpha$ the vector $(\Delta\alpha_1,\dots \Delta\alpha_k)$ the posterior distribution over $\vtheta$ turns out to be
\[
  \Pr(\vtheta|\dataobs)=\dirichlet(\valpha+\Delta \valpha). 
\]
where $+$ denotes the componentwise sum of vectors of reals. 

\noindent \textbf{Differential Privacy.} 
\begin{definition}
$\epsilon-$differential privacy.

A randomized mechanism $\mathcal{M}: \mathcal{X} \rightarrow \mathcal{Y}$ is differential privacy, iff for any adjacent input $\dataobs, \dataobs' \in \mathcal{X}$, a metric $H$ over $\mathcal{Y}$ and a $B \subseteq H(\mathcal{Y})$, $\mathcal{M}$ satisfies:
\begin{equation*}
\mathbb{P}[H(\mathcal{M}(\dataobs)) \in B] = e^\epsilon \mathbb{P}[H(\mathcal{M}(\dataobs')) \in B],
\end{equation*}
where $\dataobs = (x_i)_{i = 1}^n$ and $\dataobs = (x'_i)_{i = 1}^n$ is adjacent if there is only one $j$ that $x_j \neq x'_j$ and $x_i = x'_i$ for $i = 1, 2, \cdots, n; i \neq j$. 
\end{definition}

\begin{definition}
$(\epsilon,\delta)-$differential privacy.

A randomized mechanism $\mathcal{M}: \mathcal{X} \rightarrow \mathcal{Y}$ is differential privacy, iff for any adjacent input $\dataobs, \dataobs' \in \mathcal{X}$, a metric $H$ over $\mathcal{Y}$ and a $B \subseteq H(\mathcal{Y})$, $\mathcal{M}$ satisfies:
\begin{equation*}
\mathbb{P}[H(\mathcal{M}(\dataobs)) \in B] = e^\epsilon \mathbb{P}[H(\mathcal{M}(\dataobs')) \in B] + \delta,
\end{equation*}
vwhere $\dataobs = (x_i)_{i = 1}^n$ and $\dataobs = (x'_i)_{i = 1}^n$ is adjacent if there is only one $j$ that $x_j \neq x'_j$ and $x_i = x'_i$ for $i = 1, 2, \cdots, n; i \neq j$. 
\end{definition}


\section{Mechanism Proposition}
\label{sec_mechs}
Given a prior distribution $\bprior=\betad(\alpha, \beta)$ and a sequence of $n$ observations $\dataobs\in\{0,1\}^n$, we define the follwing set:
\[
  \candidateset\equiv\{\betad(\alpha',\beta')\mid \alpha'=\alpha+\Delta\alpha, \beta'=\beta+n-\Delta\alpha\},
\]
where $\Delta\alpha$ is as defined in Section\ref{sec_background}.
Notice that $\candidateset$ has $n + 1$ elements, and
the Bayesian Inference process will produce an element from $\candidateset$
that we denote by $\bysinfer(\dataobs)$ -- we don't explicitely
parametrize the result by the prior, which from now on we consider
fixed and we denote it by $\bprior$.

\subsection{$\hexpmech$: Smoothed Hellinger Distance Based Exponential Mechanism}
\label{subsec_hexpmech}

\begin{definition}
\label{def_smoo}
The mechanism $\hexpmech(x)$ outputs a candidate $r \in \candidateset$ with probability
\begin{equation*}
\underset{z \thicksim \hexpmech}{\Pr}[z=r] = \frac {exp\big(\frac{-\epsilon\cdot\hellinger(\bysinfer(\dataobs),r)}{2\cdot S(\dataobs)}\big)}
{\displaystyle\sum_{r\in\candidateset} exp\Big(\frac{-\epsilon\cdot\hellinger(\bysinfer(\dataobs),r)}{2\cdot S(\dataobs)}\Big)}.
\end{equation*}
where $S_\beta(x)$ is the smooth sensitivity of $\hellinger(\bysinfer(x),-)$, calculated by:

\begin{equation}
  \label{eq:smooth}
   S(\dataobs)=\max_{\dataobs' \in \{0,1\}^{n}}\bigg \{LS(\dataobs') \cdot e^{-\gamma\cdot d(\dataobs,\dataobs')}\bigg\},
\end{equation}
where $d$ is the Hamming distance between two datasets, and $\beta =
\beta(\epsilon, \delta)$ is a function of $\epsilon$ and $\delta$. 
\end{definition}

This mechanism is based on the basic exponential mechanism
\cite{talwar}, with $\candidateset$ as the range and
$\hellinger(\cdot,\cdot)$ as the scoring function. The difference is
that in this mechanism we don't calibrate the noise w.r.t. to the
global sensitivity of the scoring function but w.r.t. to the smooth
sensitivity $S(\dataobs)$ -- defined by \citet{nissim2007smooth}-- of
$\hellinger(\bysinfer(\dataobs), \cdot)$.

$\gamma = \gamma(\epsilon, \delta)$ is a function of $\epsilon$ and $\delta$ to
be determined later, and where $LS(\dataobs')$ denotes the local
sensitivity at $\bysinfer(\dataobs')$, or equivalently at $\dataobs'$,
of the scoring function used in our mechanism.

This mechanism also extends to the Dirichlet-multinomial system $\dirichlet(\valpha)$ by rewriting the Hellinger distance as:
\[
  \hellinger(\dirichlet(\valpha_1), \dirichlet(\valpha_2)) = \sqrt{1 - \frac{\betaf(\frac{\valpha_1 + \valpha_2}{2})}{\sqrt{\betaf(\valpha_1) \betaf(\valpha_2)}}},
\]
and by replacing the $\candidateset$ with set of posterior Dirichlet
distributions candidates. Also, the smooth sensitivity $S(\dataobs)$
in (\ref{eq:smooth}) will be computed by letting $\dataobs'$ range
over all the elements in $\datauni^{n}$ adjacent to $\dataobs$. Notice
that $\candidateset$ has $\binom{n + 1}{m - 1}$ elements in this case. We
will denote by $\hexpmechd$ the mechanism for the
Dirichlet-multinomial system.

By setting the $\gamma$ as $\ln(1 - \frac{\epsilon}{2 \ln (\frac{\delta}{2 (n + 1)})})$ (or $\ln(1 - \frac{\epsilon}{2 \ln (\frac{\delta}{2 |\candidateset|})})$ when generalized to Dirichlet-multinomial System), $\hexpmech$ is $(\epsilon, \delta) -$differentially private.


\section{Privacy Analysis}

\subsection{Privacy Analysis for $\hexpmech$}

The differential privacy property of $\hexpmech$ is proved based on the holds of the two properties: \emph{sliding property} and \emph{dilation property}.\\

\noindent \textbf{Sliding Property of $\hexpmech$}
\begin{lem}
Given $\hexpmech(x)$ calibrated on the smooth sensitivity. Let $\lambda = f(\epsilon,
\delta)$, $\epsilon\geq 0$ and $|\delta| < 1$. Then, the following \emph{sliding property} holds:
\begin{equation*}
\underset{r \thicksim \hexpmech(x)}{Pr}[u(r,x) = \hat{s}]
\leq
e^{\frac{\epsilon}{2}} \underset{r \thicksim \hexpmech(x)}{Pr}[u(r,x) = (\Delta + \hat{s})] + \frac{\delta}{2},
\end{equation*}

\end{lem}

\begin{proof}

In what follows, we will use a correspondence between the probability
 $\underset{z \thicksim \hexpmech(x)}{Pr}[z = r]$ of every
 $r\in\candidateset$ and the probability 
 $\underset{z \thicksim \hexpmech(x)}{Pr}[\hellinger(\bysinfer(x),z) =
 \hellinger(\bysinfer(x),r)]$ for the utility score for $r$. In particular, for every
 $r\in\candidateset$ we have:

$$
\underset{z \thicksim \hexpmech(x)}{Pr}[z = r]=
\frac{1}{2}\Big (\underset{z \thicksim \hexpmech(x)}{Pr}[\hellinger(\bysinfer(x),z) =
 \hellinger(\bysinfer(x),r)]\Big )
$$

To see this, it is enough to notice that: $\underset{z \thicksim \hexpmech(x)}{Pr}[z = r]$ is proportional too $\hellinger(\bysinfer(x),r)$, i.e., $u(x,z)$. We can derive, if $u(r,x) = u(r',x)$ then $\underset{z \thicksim \hexpmech(x)}{Pr}[z = r] = \underset{z \thicksim \hexpmech(x)}{Pr}[z = r']$. We assume the number of candidates $z \in \mathcal{R}$ that satisfy $u(z,x) = u(r,x)$ is $|r|$, we have  $\underset{z \thicksim \hexpmech(x)}{Pr}[u(z,x) = u(r,x)] = |r| \underset{z \thicksim \hexpmech(x)}{Pr}[z = r]$. Because Hellinger distance  $\hellinger(\bysinfer(x),z)$ is axial symmetry, where the $\bysinfer(x)$ is the symmetry axis. It can be infer that $|z| = 2$ for any candidates, apart from the true output, i.e., $\underset{z \thicksim \hexpmech(x)}{Pr}[u(z,x) = u(r,x)] = 2 \underset{z \thicksim \hexpmech(x)}{Pr}[z = r]$. This parameter can be eliminate in both sides in proof.

We denote the normalizer of the probability mass in $\hexpmech(x)$: $\sum_{r' \in \mathcal{R}}exp(\frac{\epsilon u(r',x)}{2 S(x)})$ as $NL(x)$:
\begin{equation*}
\begin{split}
LHS 
  = \underset{r \thicksim \hexpmech(x)}{Pr}[u(r,x) = \hat{s}]
& = \frac{exp(\frac{\epsilon \hat{s}}{2 S(x)})}{NL(x)}\\
& = \frac{exp(\frac{\epsilon (\hat{s} + \Delta - \Delta)}{2 S(x)})}{NL(x)}\\
& = \frac{exp(\frac{\epsilon (\hat{s} + \Delta)}{2 S(x)} + \frac{- \epsilon \Delta}{2 S(x)})}{NL(x)}\\
& = \frac{exp(\frac{\epsilon (\hat{s} + \Delta)}{2 S(x)})}{NL(x)} \cdot e^{\frac{- \epsilon \Delta}{2 S(x)})}.\\
\end{split}
\end{equation*}

By bounding the $\Delta \geq -S(x)$, we can get:

\begin{equation*}
\begin{split}
\frac{exp(\frac{\epsilon (\hat{s} + \Delta)}{2 S(x)})}{NL(x)} \cdot e^{\frac{- \epsilon \Delta}{2 S(x)}}
& \leq \frac{exp(\frac{\epsilon (\hat{s} + \Delta)}{2 S(x)})}{NL(x)} \cdot e^{\frac{\epsilon}{2}}\\
&  =  e^{\frac{\epsilon}{2}} \underset{z \thicksim \hexpmech(x)}{Pr}[u(r,x) = (\Delta + \hat{s})] \leq RHS\\
\end{split}
\end{equation*}

\end{proof}



\noindent \textbf{Dilation Property of $\hexpmech$}
\begin{lem}
for any exponential mechanism $\hexpmech(x)$, $\lambda < |\beta|$, $\epsilon$, $|\delta| < 1$ and $\beta \leq \ln(1 - \frac{\epsilon}{2 \ln (\frac{\delta}{2 (n + 1)})})$, the dilation property holds:

\begin{equation*}
\underset{z \thicksim \hexpmech(x)}{Pr}[u(z,x) = c]
\leq
e^{\frac{\epsilon}{2}} \underset{z \thicksim \hexpmech(x)}{Pr}[u(z,x) = e^{\lambda} c] + \frac{\delta}{2},
\end{equation*}
where the sensitivity in mechanism is still smooth sensitivity as above.
\end{lem}

\begin{proof}

The sensitivity is always greater than 0, and our utility function $-\hellinger(\bysinfer(x),z)$ is smaller than zero, i.e., $u(z,x) \leq 0$, we need to consider two cases where $\lambda < 0$, and $\lambda > 0$:

We set the $h(c) = Pr[u(\hexpmech(x)) = c] = 2\frac{exp(\frac{\epsilon z}{2 S(x)})}{NL(x)}$.

We first consider $\lambda < 0$. In this case, $1 < e ^ {\lambda}$, so the ratio $\frac{h(c)}{h(e^{\lambda}c)} = \frac{exp(\frac{\epsilon c}{2 S(x)})}{exp(\frac{\epsilon (c \cdot e^{\lambda})}{2 S(x)})}$ is at most $\frac{\epsilon}{2}$.

Next, we proof the dilation property for $\lambda > 0$, The ratio of $\frac{h(c)}{h(e^{\lambda}c)}$ is $\exp(\frac{\epsilon}{2} \cdot \frac{u(\hexpmech(x)) (1 - e^{\lambda})}{S(x)})$. Consider the event $G = \{ \hexpmech(x) : u(\hexpmech(x)) \leq \frac{S(x)}{(1 - e^{\lambda})}\}$. Under this event, the log-ratio above is at most $\frac{\epsilon}{2}$. The probability of $G$ under density $h(c)$ is $1 - \frac{\delta}{2}$. Thus, the probability of a given event $z$ is at most $Pr[c \cap G] + Pr[\overline{G}] \leq e^{\frac{\epsilon}{2}} Pr[e^{\lambda}c \cap G] + \frac{\delta}{2} \leq e^{\frac{\epsilon}{2}} Pr[e^{\lambda}c] + \frac{\delta}{2}$.\\


\textbf{Detail proof:}
	
	By simplification, we get this formula: $u(\hexpmech(x)) \leq \frac{S(x)}{(1 - e ^ {\lambda})}$
\begin{itemize}
	\item $\lambda < 0$

		The left hand side will always be smaller than 0 and the right hand side greater than 0. This will always holds, i.e.
		\begin{equation*}
		u(\hexpmech(x)) \leq \frac{S(x)}{(1 - e ^ {\lambda})}
		\end{equation*}
		is always true when $\lambda < 0$
	\item $\lambda > 0$


		Because $\hat{s} = u(r)$ where $r \thicksim \hexpmech(x)$, we can substitute $\hat{s}$ with $u(\hexpmech(x))$. Then, what we need to proof under the case $\lambda > 0$ is:
		\begin{equation}
		\label{eq_dilation_case2}
		u(\hexpmech(x)) \leq \frac{S(x)}{(1 - e ^ {\lambda})}
		\end{equation}
		Based on the accuracy property of exponential mechanism:
		\begin{equation*}
		\begin{split}
		Pr[u(\expmech{}{x}{u}{\candidateset}) \leq c] 
		& \leq \frac{|\mathcal{R}|exp(\frac{\epsilon c}{2 GS})}{|\mathcal{R}_{OPT}| exp(\frac{\epsilon OPT_{u(x)}}{2 GS})}\\
		\end{split}
		\end{equation*}

		we derived the accuracy bound for $\hexpmech$:

		\begin{equation*}
		\begin{split}
		Pr[u(\hexpmech(x)) \leq c] 
		& \leq |\candidateset|exp(\frac{\epsilon c}{2 S(x)})
		\end{split}
		\end{equation*}

		In Beta-binomial system, $|\candidateset| = n + 1$, apply this bound to eq. \ref{eq_dilation_case2}:
		\begin{equation*}
		\begin{split}
		Pr[u(\hexpmech(x)) \leq \frac{S(x)}{(1 - e ^ {\lambda})}] 
		& = (n + 1)exp(\frac{\epsilon S(x)}{(1 - e ^ {\lambda})}/2 S(x))\\
		& = (n + 1)exp(\frac{\epsilon}{2 (1 - e ^ {\lambda})})\\
		\end{split}
		\end{equation*}

		When we set $\lambda \leq \ln(1 - \frac{\epsilon}{2 \ln (\frac{\delta}{2 (n + 1)})})$, it is easily to derive that $Pr[u(\hexpmech(x)) \leq \frac{S(x)}{(1 - e ^ {\lambda})}] \leq \frac{\delta}{2}$.

		In Dirichlet-multinomial system, $\lambda$ is set as $ \leq \ln(1 - \frac{\epsilon}{2 \ln (\frac{\delta}{2 |\candidateset|})})$ since $|\candidateset| \neq n + 1$ any more.

\end{itemize}

\end{proof}

\noindent \textbf{$(\epsilon, \delta)-$Differential Privacy of $\hexpmech$}
\begin{lem}
\label{lem_hexpmech_privacy}
$\hexpmech$ is $(\epsilon, \delta)$-differential privacy.
\end{lem}

\begin{proof}
of Lemma \ref{lem_hexpmech_privacy}: For all neighboring $x, y \in D^n$ and all sets $\mathcal{S}$, we need to show that:
\begin{equation*}
\underset{z \thicksim \hexpmech(x)}{Pr}[ z \in \mathcal{S}] \leq e^{\epsilon} \underset{z \thicksim \hexpmech(y)}{Pr}[z \in \mathcal{S}] + \delta. 
\end{equation*}
Given that $2\Big( \underset{z \thicksim \hexpmech(x)}{Pr}[ z \in \mathcal{S}]\Big) = \underset{z \thicksim \hexpmech(x)}{Pr}[ u(x,z) \in \mathcal{U}]$, let $\mathcal{U}_1 = \frac{u(y,z) - u(x,z)}{S(x)}$, $\mathcal{U}_2 = \mathcal{U} + \mathcal{U}_1$ and $\mathcal{U}_3 = \mathcal{U}_2 \cdot \frac{S(x)}{S(y)} \cdot \ln(\frac{NL(x)}{NL(y)})$. Then,

\begin{equation*}
\begin{split}
2\Big( \underset{z \thicksim \hexpmech(x)}{Pr}[ z \in \mathcal{S}]\Big)
& = \underset{z \thicksim \hexpmech(x)}{Pr}[ u(x,z) \in \mathcal{U}]\\
& \leq e^{\epsilon / 2} \cdot \underset{z \thicksim \hexpmech(x)}{Pr}[ u(x,z) \in \mathcal{U}_2]\\
& \leq e^{\epsilon} \cdot \underset{z \thicksim \hexpmech(x)}{Pr}[ u(x,z) \in \mathcal{U}_3] + e^{\epsilon/2} \cdot \frac{\delta'}{2}\\
& = e^{\epsilon} \cdot \underset{z \thicksim \hexpmech(y)}{Pr}[ u(y,z) \in \mathcal{U}] + \delta = 2\Big( e^{\epsilon} \cdot \underset{z \thicksim \hexpmech(x)}{Pr}[ z \in \mathcal{S}] \Big) + \delta\\
\end{split}
\end{equation*}

The first inequality holds by the sliding property, since the $\mathcal{U}_1 \geq -S(x)$. The second inequality holds by the dilation property, since $\frac{S(x)}{S(y)} \cdot \ln(\frac{NL(x)}{NL(y)}) \leq 1 - \frac{\epsilon}{2 \ln (\frac{\delta}{2 (n + 1)})}$.

\end{proof}

\section{Accuracy Analysis}

\subsection{Accuracy Bound for Baseline Mechanisms}

\subsection{Accuracy Bound for $\hexpmech$}
\label{subsec_accuracy_smoo}
We explored three accuracy bounds for our exponential mechanism with smooth sensitivity.

First is the tight bound with very accurate calculation.
\begin{equation*}
\underset{z \thicksim \hexpmech(x)}{Pr}[ \hellinger(\bysinfer(x), z) \geq c] = \sum\limits_{\{z | \hellinger(\bysinfer(x), z) \geq c\}} \frac{e^{\frac{- \epsilon \hellinger(\bysinfer(x), z)}{S(x)}}}{NL_x}.
\end{equation*}

In order to be more efficient, we designed the second accuracy bound which is slightly looser than the first one:

\begin{equation*}
\underset{z \thicksim \hexpmech(x)}{Pr}[ \hellinger(\bysinfer(x), z) \geq c] \leq \frac{|R| \exp{(\frac{- \epsilon c}{S(x)}})}{NL_x}.
\end{equation*}

In the second bound, we still need to calculate the normaliser every time. So we want make further improvements on efficiency like follows:

\begin{equation*}
\underset{z \thicksim \hexpmech(x)}{Pr}[ \hellinger(\bysinfer(x), z) \geq c] \leq \ \frac{|R| \exp{(\frac{- \epsilon c}{S(x)}})}{N(n)},
\end{equation*}
where we replace the $NL_x$ with a value only related to the size of the data. However, we haven't figured out the formula of this $N(n)$.

Moreover, based on the accuracy bound in Sec. \ref{subsec_accuracy_global}, we can derive a loose bound:

\begin{equation*}
\begin{split}
Pr[u(\hexpmech(x)) \leq c] 
& \leq |\candidateset|exp(\frac{\epsilon c}{2 S(x)}),
\end{split}
\end{equation*}

which has been used in the dilation property proof.





\bibliographystyle{ACM-Reference-Format}
\bibliography{bayesian.bib}

\end{document}

