% \documentclass[sigconf]{acmart}
\documentclass{article}

% We want page numbers on submissions

%%\ccsPaper{9999} % TODO: replace with your paper number once obtained
\usepackage{amsmath,amsfonts,amssymb,amsthm,epsfig,epstopdf,titling,url,array}
\newcommand{\bysinfer}{\mathsf{BI}}
\newcommand{\hlg}{\mathsf{H}}
\newcommand{\betafun}{\mathsf{beta}}
%\newcommand{\theHalgorithm}{\arabic{algorithm}}
\theoremstyle{plain}
\newtheorem{thm}{Theorem}[section]
\newtheorem{lem}[thm]{Lemma}
\newtheorem{prop}[thm]{Proposition}
\newtheorem*{cor}{Corollary}

\theoremstyle{definition}
\newtheorem{defn}{Definition}[section]
\newtheorem{conj}{Conjecture}[section]
\newtheorem{exmp}{Example}[section]

\theoremstyle{remark}
\newtheorem*{rem}{Remark}
\newtheorem*{note}{Note}

\usepackage{accents}

\begin{document}
\title{Proof of Sliding and Dilation Property}

\author[*]{Jiawen Liu}
\author[**]{Mark Bun}
\author[*]{Gian Pietro Farina}
\author[*]{Marco Gaboardi}
\affil[*]{Department of Computer Science and Engineering, University at Buffalo, SUNY. \{jliu223,gianpiet,gaboardi\}@buffalo.edu}
\affil[**]{Department of Computer Science, Princeton University. {mbun@cs.princeton.edu}}
\date{}
\maketitle



\section{Preliminaries}
\label{sec_background}

\noindent \textbf{Bayesian Inference.}

Given a prior belief $\Pr(\theta)$ on some parameter $\theta$,
and an observation $\dataobs$, the posterior distribution on $\theta$ given $\dataobs$ is computed as:
\[
  \Pr(\theta | \dataobs) = \frac{\Pr(\dataobs | \theta) \cdot \Pr(\theta)}{\Pr(\dataobs)}
\]
where the expression $\Pr(\dataobs | \theta)$ denotes the
\emph{likelihood} of observing $\dataobs$ under a value of
$\theta$. Since we consider $\dataobs$ to be fixed, the likelihood is
a function of $\theta$.
For the same reason $\Pr(\dataobs)$ is a constant independent of $\theta$.
Usually in statistics the prior distribution $\Pr(\theta)$ is chosen so that it represents
the initial belief on $\theta$, that is, when no data has been observed. In practice though,
prior distributions and likelihood functions are usually chosen so that the posterior
belongs to the same \emph{family} of distributions. In this case we say that the prior
is conjugate to the likelihood function. Use of a conjugate prior
simplifies calculations and allows for inference to be performed in a
recursive fashion over the data.


\noindent \textbf{Beta-binomial System.}

In this work we will consider a specific instance of Bayesian inference and one of its generalizations.
specifically, a Beta-binomial mode. We will consider the situation the underlying data is binomial distribution ($\thicksim binomial(\theta)$), where $\theta$ represents
the parameter --informally called \emph{bias}-- of a Bernoulli
distributed random variable. The
prior distribution over $\theta\in [0,1]$ is going to be a beta
distribution, $\betad(\alpha, \beta)$, with parameters
$\alpha,\beta\in\mathbb{R}^{+}$, and with p.d.f:
\[
  \Pr(\theta)\equiv \frac{\theta^{\alpha} (1- \theta)^{\beta}}{\betaf(\alpha,\beta)}
\]
where $\betaf(\cdot,\cdot)$ is the beta function.
The data $\dataobs$ will be a sequence of $n\in\mathbb{N}$ binary values, that is $\dataobs= (x_1,\dots x_n), x_i\in\{0,1\}$, and the likelihood function is:
\[
  \Pr(\dataobs | \theta)\equiv \theta^{\Delta \alpha}(1-\theta)^{n - \Delta \alpha}
\]
where $\Delta \alpha = \displaystyle\sum_{i=1}^{n}x_i$.
From this it can easily be derived that the posterior distribution is:
\[
  \Pr(\theta|\dataobs)=\betad(\alpha + \Delta \alpha,\beta + n - \Delta \alpha)
\]


\noindent \textbf{Dirichlet-multinomial Systems.}

The beta-binomial model can be immediately generalized to Dirichlet-multinomial, with underlying data multinomially distributed. The \emph{bias} is represented by parameter $\vtheta$, the vector of parameters of a categorically distributed random variable. The prior distribution over $\vtheta\in [0,1]^{k}$
is given by a Dirichelet distribution, $\dirichlet(\valpha)$, for $k\in\mathbb{N}$,
and $\valpha\in(\mathbb{R}^{+})^{k}$, with p.d.f:
\[
  \Pr(\vtheta)\equiv\frac{1}{\mbetaf(\valpha)}\cdot \displaystyle\prod_{i=1}^{k}{\theta_i^{\alpha_i-1}}
\]
where $\mbetaf(\cdot)$ is the generalized beta function.
The data $\dataobs$ will be a sequence of $n\in\mathbb{N}$ values
coming from a universe $\datauni$, such that $\mid\datauni \mid=k$.
The likelihood function will be:
\[
  \Pr(\dataobs|\vtheta)\equiv \displaystyle\prod_{a_i\in\datauni}\theta_{i}^{\Delta \alpha_i},
\]
with $\Delta \alpha_i=\displaystyle\sum_{j=1}^{n}\iverson{x_j=a_i}$, where $\iverson{\cdot}$ represents Iverson bracket notation.
Denoting by $\Delta\valpha$ the vector $(\Delta\alpha_1,\dots \Delta\alpha_k)$ the posterior distribution over $\vtheta$ turns out to be
\[
  \Pr(\vtheta|\dataobs)=\dirichlet(\valpha+\Delta \valpha). 
\]
where $+$ denotes the componentwise sum of vectors of reals. 

\noindent \textbf{Differential Privacy.} 
\begin{definition}
\label{def_epsilon_dp}
$\epsilon-$differential privacy.

A randomized mechanism $\mathcal{M}: \mathcal{X} \rightarrow \mathcal{Y}$ is differential privacy, iff for any adjacent\footnote{Given $\dataobs, \dataobs'$  we say that $\dataobs$ and $\dataobs'$ are adjacent and we write, $\adj{\dataobs}{\dataobs'}$, iff\\
$\displaystyle \sum_{i}^{n}\iverson{x_i = x'_i }\leq 1$. } input $\dataobs, \dataobs' \in \mathcal{X}$, a metric $H$ over $\mathcal{Y}$ and a $B \subseteq H(\mathcal{Y})$, $\mathcal{M}$ satisfies:
\begin{equation*}
\Pr[H(\mathcal{M}(\dataobs)) \in B] = e^\epsilon \Pr[H(\mathcal{M}(\dataobs')) \in B].
\end{equation*}

\end{definition}

\begin{definition}
\label{def_epsilon_delta_dp}
$(\epsilon,\delta)-$differential privacy.

A randomized mechanism $\mathcal{M}: \mathcal{X} \rightarrow \mathcal{Y}$ is differential privacy, iff for any $\adj{\dataobs}{\dataobs'} \in \mathcal{X}$, a metric $H$ over $\mathcal{Y}$ and a $B \subseteq H(\mathcal{Y})$, $\mathcal{M}$ satisfies:
\begin{equation*}
\Pr[H(\mathcal{M}(\dataobs)) \in B] = e^\epsilon \Pr[H(\mathcal{M}(\dataobs')) \in B] + \delta,
\end{equation*}
vwhere $\dataobs = (x_i)_{i = 1}^n$ and $\dataobs = (x'_i)_{i = 1}^n$ is adjacent if there is only one $j$ that $x_j \neq x'_j$ and $x_i = x'_i$ for $i = 1, 2, \cdots, n; i \neq j$. 
\end{definition}


\subsection{$\hexpmech$: Smoothed Hellinger Distance Based Exponential Mechanism}
\label{subsec_hexpmech}

\begin{definition}
\label{def_smoo}
The mechanism $\hexpmech(\dataobs)$ outputs a candidate $r \in \candidateset$ with probability
\begin{equation*}
\underset{z \thicksim \hexpmech}{\Pr}[z=r] = \frac {exp\big(\frac{-\epsilon\cdot\hellinger(\bysinfer(\dataobs),r)}{2\cdot S(\dataobs)}\big)}
{\nomalizer{\dataobs}}.
\end{equation*}
where $S_\beta(\dataobs)$ is the smooth sensitivity of $\hellinger(\bysinfer(\dataobs),-)$, calculated by:

\begin{equation}
  \label{eq:smooth}
   S(\dataobs)=\max_{\dataobs' \in \{0,1\}^{n}}\bigg \{LS(\dataobs') \cdot e^{-\gamma\cdot d(\dataobs,\dataobs')}\bigg\},
\end{equation}
where $d$ is the Hamming distance between two datasets, and $\gamma$ is the smooth upper bound in smooth sensitivity.
\end{definition}

This mechanism is based on the standard exponential mechanism
\cite{talwar}, with $\candidateset$ as the range and
$\hellinger(\cdot,\cdot)$ as the scoring function. The difference is
that in this mechanism we don't calibrate the noise w.r.t. to the
global sensitivity of the scoring function but w.r.t. to the smooth
sensitivity $S(\dataobs)$ -- defined by \cite{nissim2007smooth}-- of
$\hellinger(\bysinfer(\dataobs), \cdot)$, where $LS(\dataobs')$ denotes the local
sensitivity at $\dataobs'$, of the scoring function used in our mechanism and  $\gamma$ is the smooth upper bound for smooth sensitivity.

This mechanism also extends to the Dirichlet-multinomial system $\dirichlet(\valpha)$ by rewriting the Hellinger distance as:
\[
  \hellinger(\dirichlet(\valpha_1), \dirichlet(\valpha_2)) = \sqrt{1 - \frac{\betaf(\frac{\valpha_1 + \valpha_2}{2})}{\sqrt{\betaf(\valpha_1) \betaf(\valpha_2)}}},
\]
and by replacing the $\candidateset$ with set of posterior Dirichlet
distributions candidates. Also, the smooth sensitivity $S(\dataobs)$
in (\ref{eq:smooth}) will be computed by letting $\dataobs'$ range
over all the elements in $\datauni^{n}$ adjacent to $\dataobs$. Notice
that $\candidateset$ has $\binom{n + 1}{m - 1}$ elements in this case. We
will denote by $\hexpmechd$ the mechanism for the
Dirichlet-multinomial system.

% By setting the $\gamma$ as $\ln(1 - \frac{\epsilon}{2 \ln (\frac{\delta'}{2 (n + 1)})})$, $\hexpmech$ is $(\epsilon, \delta) -$differentially private, where $\delta = \frac{e^{\frac{\epsilon}{2}} \delta'}{2}$.


\section{Privacy Analysis}

\subsection{Privacy Analysis for Baseline Mechanisms}
In baseline mechanisms, \emph{exponential mechanism}, \emph{Laplace mechanism}, \emph{improved Laplace mechanism} are $\epsilon-$differential privacy provided by \cite{dwork2014algorithmic}. The \emph{exponential mechanism with local sensitivity} is non-differential privacy, also from \cite{dwork2014algorithmic}.

\subsection{Privacy Analysis for $\hexpmech$}

The differential privacy property of $\hexpmech$ is proved based on following two lemmas: Lemma \ref{lem_sliding} and Lemma \ref{lem_dilation}.\\


In what follows, let $ \ux{r} = \hellinger(\bysinfer(\dataobs), r)$ for $r \in \candidateset$, we first introduce Lemma \ref{lem_score_pro_eq} and \ref{lem_score_pro_convert}.

\begin{lem}
\label{lem_score_pro_eq}
  Given $\hexpmech(\dataobs)$ and arbitrary $r_1, r_2 \in \candidateset$, then following holds:
  $$
  \ux{r_1} = \ux{r_2} \iff \hexpmechPr{\dataobs}{z = r_1} = \hexpmechPr{\dataobs}{z = r_2}.
  $$  
\end{lem}
\begin{proof} of Lemma \ref{lem_score_pro_eq}.
\begin{itemize}
  \item $\implies$

  $\ux{r_1} = \ux{r_2}$

  Apply $\ux{r} = \hellinger(\bysinfer(\dataobs), r)$ for $r \in \candidateset$:\\
  $\implies \hellinger(\bysinfer(\dataobs), r_1) = \hellinger(\bysinfer(\dataobs), r_2)$. \\
  $\implies \frac {exp\big(\frac{-\epsilon\cdot\hellinger(\bysinfer(\dataobs),r_1)}{2\cdot S(\dataobs)}\big)}{\nomalizer{\dataobs}} 
  = \frac {exp\big(\frac{-\epsilon\cdot\hellinger(\bysinfer(\dataobs),r_2)}{2\cdot S(\dataobs)}\big)}{\nomalizer{\dataobs}}$.

  Apply\ Definition\ \ref{def_smoo}:\\
  $\implies \hexpmechPr{\dataobs}{z = r_1} = \hexpmechPr{\dataobs}{z = r_2}$.


  \item $\impliedby$

  $\hexpmechPr{\dataobs}{z = r_1} = \hexpmechPr{\dataobs}{z = r_2} $

  Apply Definition \ref{def_smoo}:\\
  $\implies \frac {exp\big(\frac{-\epsilon\cdot\hellinger(\bysinfer(\dataobs),r_1)}{2\cdot S(\dataobs)}\big)}{\nomalizer{\dataobs}} 
    = \frac {exp\big(\frac{-\epsilon\cdot\hellinger(\bysinfer(\dataobs),r_2)}{2\cdot S(\dataobs)}\big)}{\nomalizer{\dataobs}}$.\\
  $\implies  \hellinger(\bysinfer(\dataobs), r_1) = \hellinger(\bysinfer(\dataobs), r_2)$.

  Apply\ $\ux{r} = \hellinger(\bysinfer(\dataobs), r)$ for $r \in \candidateset$:
  $\implies \ux{r_1} = \ux{r_2}$.

\end{itemize}

\end{proof}

\begin{definition}
\label{def_cardinality}
  $\cardinality{\dataobs}{r}$, $R^{\dataobs}_r$.

  Given $\hexpmech(\dataobs)$, define $R^{\dataobs}_r \subset \candidateset$, s.t. $r' \in R^{\dataobs}_r \iff \ux{r'} = \ux{r}$.

  Given $\hexpmech(\dataobs)$, define $\cardinality{\dataobs}{r}$ as the cardinality of $R^{\dataobs}_r$. 
\end{definition}

\begin{lem}
\label{lem_score_pro_convert}
 Given $\hexpmech(\dataobs)$, then the following property holds:
\begin{equation*}
\hexpmechPr{\dataobs}{\ux{z} = \ux{r}} = \cardinality{\dataobs}{r}\hexpmechPr{\dataobs}{z = r}
 \end{equation*}
\end{lem}
\begin{proof} of Lemma \ref{lem_score_pro_convert}.\\
To compute $\hexpmechPr{\dataobs}{\ux{z} = \ux{r}}$\\
$\iff $ to compute $\hexpmechPr{\dataobs}{ z \in R | R = \{r' | r' \in \candidateset \land \ux{r'} = \ux{r}\}}.$\\
By Definition \ref{def_cardinality}, $R^{\dataobs}_r$ is this set.\\
$\implies \hexpmechPr{\dataobs}{\ux{z} = \ux{r}} = \hexpmechPr{\dataobs}{z \in R^{\dataobs}_r} $.\\
By the independence of $r' \in R_r$, $ \hexpmechPr{\dataobs}{z \in R^{\dataobs}_r}  =\sum\limits_{r'\in R^{\dataobs}_r} \hexpmechPr{\dataobs}{z = r'}$ ,\\
Apply Lemma \ref{lem_score_pro_eq} in $R^{\dataobs}_r$, \\
$\implies$ for any $r', r''\in R^{\dataobs}_r$, $\hexpmechPr{\dataobs}{z = r'} = \hexpmechPr{\dataobs}{z = r''}$.\\
$\implies$ $\sum\limits_{r'\in R^{\dataobs}_r} \hexpmechPr{\dataobs}{z = r'} = \cardinality{\dataobs}{r}\hexpmechPr{\dataobs}{z = r}$.\\
$\implies$ $\hexpmechPr{\dataobs}{\ux{z} = \ux{r}} = \cardinality{\dataobs}{r}\hexpmechPr{\dataobs}{z = r}$.\\
\end{proof}


\subsubsection{Sliding Property of $\hexpmech$}


\begin{lem}
\label{lem_sliding}

Given $\hexpmech(\dataobs)$ calibrated on the smooth sensitivity. In Beta-binomial system, for any $r\in \candidateset$, let $\epsilon\geq 0$ and $\ux{r} + k \geq 0$, the following \emph{sliding property} holds:
\begin{equation*}
\hexpmechPr{\dataobs}{\ux{z} = \ux{r}}
\leq
e^{(\frac{\epsilon k}{2 S(\dataobs)} + \ln 2)} \hexpmechPr{\dataobs}{\ux{z} = ( \ux{r} + k)}.
\end{equation*}
\end{lem}

\begin{proof} of Lemma \ref{lem_sliding}:

In Beta-binomial system, $\cardinality{\dataobs}{r}$ takes two possible values: $1$, $2$:
\begin{itemize}
  \item[-] $\cardinality{\dataobs}{r} = 1$, $\ux{r} = 0$.  $R^{\dataobs}_r$ contains only the correct posterior.
  \item[-] $\cardinality{\dataobs}{r} = 2$, $\ux{r} \neq 0$. $R^{\dataobs}_r$ contains 2 elements by the symmetry property of $\hellinger$.
\end{itemize}

\todo{
In Beta-binomial system, $\cardinality{\dataobs}{r}$ takes two possible values: $1$, $2$. Let $\mathcal{R}_1$, $\mathcal{R}_2$ be a partition of $\candidateset$:
\begin{itemize}
  \item[*] $\mathcal{R}_1 = \{r | r \in \candidateset\ s.t.\ \cardinality{\dataobs}{r} = 1\}$.
  \item[*] $\mathcal{R}_2 = \{r | r \in \candidateset\ s.t.\ \cardinality{\dataobs}{r} = 2\}$.
\end{itemize}
}

Let $\ux{t} = \ux{r} + k$. The proof is given by cases:
\begin{itemize}
  \item {\boldmath$\ux{r} \neq 0$, $\ux{r} + k \neq 0$}. \todo{$r \in \mathcal{R}_2 \land t \in \mathcal{R}_2$} \\
  By $\ux{r} + k \neq 0$, $\cardinality{\dataobs}{t} = 2$. \todo{By $t \in \mathcal{R}_2$, $\cardinality{\dataobs}{t} = 2$}\\
  By $\ux{r} \neq 0 $, $\cardinality{\dataobs}{r} = 2$. \todo{By $r \in \mathcal{R}_2$, $\cardinality{\dataobs}{r} = 2$}
  \begin{itemize}
    \item  $k \geq 0$: $\implies$
      \begin{equation*}
      \begin{split}
      LHS 
      & = \cardinality{\dataobs}{r} \cdot \hexpmechPr{\dataobs}{z = r}     (Apply\ Lemma\ \ref{lem_score_pro_convert})\\
      & = \cardinality{\dataobs}{r} \cdot \frac {exp\big(\frac{-\epsilon\cdot\hellinger(\bysinfer(\dataobs),r)}{2\cdot S(\dataobs)}\big)}{\nomalizer{\dataobs}} 
      (Apply\ Definition\ \ref{def_smoo})\\
      & = \cardinality{\dataobs}{r} \cdot \frac{exp(\frac{-\epsilon \ux{r}}{2 S(\dataobs)})}{\unomalizer{\dataobs}}
      (Apply\ \ux{r} = \hellinger(\bysinfer(\dataobs), r))\\
      & = 2 \cdot \frac{exp(\frac{-\epsilon \ux{r}}{2 S(\dataobs)})}{\unomalizer{\dataobs}}                  (Apply\ \cardinality{\dataobs}{r} = 2)\\
      & = 2 \cdot \frac{exp(\frac{-\epsilon (\ux{r} + k - k)}{2 S(\dataobs)})}{\unomalizer{\dataobs}}\\
      & = 2 \cdot \frac{exp(\frac{-\epsilon (\ux{r} + k)}{2 S(\dataobs)} + \frac{\epsilon k}{2 S(\dataobs)})}{\unomalizer{\dataobs}}\\
      & = 2 \cdot \frac{exp(\frac{-\epsilon (\ux{r} + k)}{2 S(\dataobs)})}{\unomalizer{\dataobs}} \cdot e^{\frac{\epsilon k}{2 S(\dataobs)}}\\
      & = 2 \cdot \frac{exp(\frac{-\epsilon (\ux{t})}{2 S(\dataobs)})}{\unomalizer{\dataobs}} \cdot e^{\frac{\epsilon k}{2 S(\dataobs)}}  (Apply \ \ux{t} = \ux{r} + k)\\
      & = e^{\frac{\epsilon k}{2 S(\dataobs)}} \cdot 2 \cdot \frac {exp\big(\frac{-\epsilon\cdot\hellinger(\bysinfer(\dataobs), t)}{2\cdot S(\dataobs)}\big)}{\nomalizer{\dataobs}}
      (Apply\ \ux{r} = \hellinger(\bysinfer(\dataobs), r))\\
      & = e^{\frac{\epsilon k}{2 S(\dataobs)}} \cdot 2 \cdot \hexpmechPr{\dataobs}{z = t}                                 (Apply\ Definition\ \ref{def_smoo})\\
      & = e^{\frac{\epsilon k}{2 S(\dataobs)}} \cdot \cardinality{\dataobs}{t} \cdot \hexpmechPr{\dataobs}{z = t}         (Apply \ \cardinality{\dataobs}{t} = 2)\\
      & = e^{\frac{\epsilon k}{2 S(\dataobs)}} \cdot \hexpmechPr{\dataobs}{\ux{z} = \ux{t}}                               (Apply \ Lemma\ \ref{lem_score_pro_convert})\\
      & = e^{\frac{\epsilon k}{2 S(\dataobs)}} \cdot \hexpmechPr{\dataobs}{\ux{z} = (\ux{r} + k)}                         (Apply \ \ux{t} = \ux{r} + k)\\
      & \leq e^{(\frac{\epsilon k}{2 S(\dataobs)} + \ln 2)} \cdot \hexpmechPr{\dataobs}{\ux{z} = (\ux{r} + k)}\\
      & = RHS.
      \end{split}
      \end{equation*}

    \item $-\ux{r} < k < 0$: $\implies$
      \begin{equation*}
      \begin{split}
      LHS 
      & = \cardinality{\dataobs}{r} \cdot \hexpmechPr{\dataobs}{z = r}     (Apply\ Lemma\ \ref{lem_score_pro_convert})\\
      & = \cardinality{\dataobs}{r} \cdot \frac {exp\big(\frac{-\epsilon\cdot\hellinger(\bysinfer(\dataobs),r)}{2\cdot S(\dataobs)}\big)}{\nomalizer{\dataobs}} 
      (Apply\ Definition\ \ref{def_smoo})\\
      & = \cardinality{\dataobs}{r} \cdot \frac{exp(\frac{-\epsilon \ux{r}}{2 S(\dataobs)})}{\unomalizer{\dataobs}}
      (Apply\ \ux{r} = \hellinger(\bysinfer(\dataobs), r))\\
      & = 2 \cdot \frac{exp(\frac{-\epsilon \ux{r}}{2 S(\dataobs)})}{\unomalizer{\dataobs}}                  (Apply\ \cardinality{\dataobs}{r} = 2)\\
      & = 2 \cdot \frac{exp(\frac{-\epsilon (\ux{r} + k - k)}{2 S(\dataobs)})}{\unomalizer{\dataobs}}\\
      & = 2 \cdot \frac{exp(\frac{-\epsilon (\ux{r} + k)}{2 S(\dataobs)} + \frac{\epsilon k}{2 S(\dataobs)})}{\unomalizer{\dataobs}}\\
      & = 2 \cdot \frac{exp(\frac{-\epsilon (\ux{r} + k)}{2 S(\dataobs)})}{\unomalizer{\dataobs}} \cdot e^{\frac{\epsilon k}{2 S(\dataobs)}}\\
      & = 2 \cdot \frac{exp(\frac{-\epsilon (\ux{t})}{2 S(\dataobs)})}{\unomalizer{\dataobs}} \cdot e^{\frac{\epsilon k}{2 S(\dataobs)}}  (Apply \ \ux{t} = \ux{r} + k)\\
      & = e^{\frac{\epsilon k}{2 S(\dataobs)}} \cdot 2 \cdot \frac {exp\big(\frac{-\epsilon\cdot\hellinger(\bysinfer(\dataobs), t)}{2\cdot S(\dataobs)}\big)}{\nomalizer{\dataobs}} 
      (Apply\ \ux{r} = \hellinger(\bysinfer(\dataobs), r))\\
      & = e^{\frac{\epsilon k}{2 S(\dataobs)}} \cdot 2 \cdot \hexpmechPr{\dataobs}{z = t}       (Apply\ Definition\ \ref{def_smoo})\\
      & = e^{\frac{\epsilon k}{2 S(\dataobs)}} \cdot \cardinality{\dataobs}{t} \cdot \hexpmechPr{\dataobs}{z = t}         (Apply \ \cardinality{\dataobs}{t} = 2)\\
      & = e^{\frac{\epsilon k}{2 S(\dataobs)}} \cdot \hexpmechPr{\dataobs}{\ux{z} = \ux{t}}               (Apply \ Lemma\ \ref{lem_score_pro_convert})\\
      & = e^{\frac{\epsilon k}{2 S(\dataobs)}} \cdot \hexpmechPr{\dataobs}{\ux{z} = (\ux{r} + k)}         (Apply \ \ux{t} = \ux{r} + k)\\
      & \leq e^{(\frac{\epsilon k}{2 S(\dataobs)} + \ln 2)} \cdot \hexpmechPr{\dataobs}{\ux{z} = (\ux{r} + k)}\\
      & = RHS.
      \end{split}
      \end{equation*}

  \end{itemize}

  \item {\boldmath$\ux{r} = 0, \ux{r} + k \neq 0$}. \todo{$r \in \mathcal{R}_1 \land t \in \mathcal{R}_2$}\\
        By $\ux{r} = 0$, $\cardinality{\dataobs}{r} = 1$. \todo{By $r \in \mathcal{R}_1$, $\cardinality{\dataobs}{t} = 1$}\\
        By $\ux{r} + k = k \neq 0$, $\cardinality{\dataobs}{t} = 2$. \todo{By $t \in \mathcal{R}_2$, $\cardinality{\dataobs}{t} = 2$}\\
        $\implies$
        \begin{equation*}
        \begin{split}
        LHS 
        & = \cardinality{\dataobs}{r} \cdot \hexpmechPr{\dataobs}{z = r}     (Apply\ Lemma\ \ref{lem_score_pro_convert})\\
        & = \cardinality{\dataobs}{r} \cdot \frac {exp\big(\frac{-\epsilon\cdot\hellinger(\bysinfer(\dataobs),r)}{2\cdot S(\dataobs)}\big)}{\nomalizer{\dataobs}} 
        (Apply\ Definition\ \ref{def_smoo})\\
        & = \cardinality{\dataobs}{r} \cdot \frac{exp(\frac{-\epsilon \ux{r}}{2 S(\dataobs)})}{\unomalizer{\dataobs}}
        (Apply\ \ux{r} = \hellinger(\bysinfer(\dataobs), r))\\
        & = \frac{exp(\frac{- \epsilon \ux{r}}{2 S(\dataobs)})}{\unomalizer{\dataobs}}                        (Apply \ \cardinality{\dataobs}{r} = 1) \\
        & = \frac{exp(\frac{- \epsilon (\ux{r} + k - k)}{2 S(\dataobs)})}{\unomalizer{\dataobs}}\\
        & = \frac{exp(\frac{- \epsilon (\ux{r} + k)}{2 S(\dataobs)} + \frac{\epsilon k}{2 S(\dataobs)})}{\unomalizer{\dataobs}}\\
        & = \frac{exp(\frac{-\epsilon (\ux{r} + k)}{2 S(\dataobs)})}{\unomalizer{\dataobs}} \cdot e^{\frac{\epsilon k}{2 S(\dataobs)}}\\
        & = \frac{exp(\frac{-\epsilon (\ux{t})}{2 S(\dataobs)})}{\unomalizer{\dataobs}} \cdot e^{\frac{\epsilon k}{2 S(\dataobs)}}  (Apply \ \ux{t} = \ux{r} + k)\\
        & = e^{\frac{\epsilon k}{2 S(\dataobs)}} \cdot \frac {exp\big(\frac{-\epsilon\cdot\hellinger(\bysinfer(\dataobs), t)}{2\cdot S(\dataobs)}\big)}{\nomalizer{\dataobs}} 
        (Apply\ \ux{r} = \hellinger(\bysinfer(\dataobs), r))\\
        & = e^{\frac{\epsilon k}{2 S(\dataobs)}} \cdot \hexpmechPr{\dataobs}{z = t}       (Apply\ Definition\ \ref{def_smoo})\\
        & < 2 \cdot e^{\frac{\epsilon k}{2 S(\dataobs)}} \cdot \hexpmechPr{\dataobs}{z = t}\\
        & = e^{\frac{\epsilon k}{2 S(\dataobs)}} \cdot \cardinality{\dataobs}{t} \cdot \hexpmechPr{\dataobs}{z = t}   (Apply \ \cardinality{\dataobs}{t} = 2)\\
        & = e^{\frac{\epsilon k}{2 S(\dataobs)}} \cdot \hexpmechPr{\dataobs}{\ux{z} = \ux{t}}         (Apply \ Lemma\ \ref{lem_score_pro_convert})\\
        & = e^{\frac{\epsilon k}{2 S(\dataobs)}} \cdot \hexpmechPr{\dataobs}{\ux{z} = (\ux{r} + k)}   (Apply \ \ux{t} = \ux{r} + k)\\
        & \leq e^{(\frac{\epsilon k}{2 S(\dataobs)} + \ln 2)} \cdot \hexpmechPr{\dataobs}{\ux{z} = ( \ux{r} + k)}\\
        & = RHS.
        \end{split}
        \end{equation*}

  \item {\boldmath$\ux{r} \neq 0, \ux{r} + k = 0$} \todo{$r \in \mathcal{R}_2 \land t \in \mathcal{R}_1$}\\
      By $\ux{r} + k = 0$, $\cardinality{\dataobs}{t} = 1$. \todo{By $t \in \mathcal{R}_1$, $\cardinality{\dataobs}{t} = 1$}\\
      By $\ux{r} \neq 0$, $\cardinality{\dataobs}{r} = 2$.  \todo{By $r \in \mathcal{R}_2$, $\cardinality{\dataobs}{t} = 2$}\\
      $\implies$
      \begin{equation*}
      \begin{split}
      LHS   
      & = \cardinality{\dataobs}{r} \cdot \hexpmechPr{\dataobs}{z = r}     (Apply\ Lemma\ \ref{lem_score_pro_convert})\\
      & = \cardinality{\dataobs}{r} \cdot \frac {exp\big(\frac{-\epsilon\cdot\hellinger(\bysinfer(\dataobs),r)}{2\cdot S(\dataobs)}\big)}{\nomalizer{\dataobs}} 
      (Apply\ Definition\ \ref{def_smoo})\\
      & = \cardinality{\dataobs}{r} \cdot \frac{exp(\frac{-\epsilon \ux{r}}{2 S(\dataobs)})}{\unomalizer{\dataobs}}
      (Apply\ \ux{r} = \hellinger(\bysinfer(\dataobs), r))\\
      & = 2 \cdot \frac{exp(\frac{-\epsilon \ux{r}}{2 S(\dataobs)})}{\unomalizer{\dataobs}}                  (Apply \ \cardinality{\dataobs}{r} = 2)\\
      & = 2 \cdot \frac{exp(\frac{-\epsilon (\ux{r} + k - k)}{2 S(\dataobs)})}{\unomalizer{\dataobs}}\\
      & = 2 \cdot \frac{exp(\frac{-\epsilon (\ux{r} + k)}{2 S(\dataobs)} + \frac{\epsilon k}{2 S(\dataobs)})}{\unomalizer{\dataobs}}\\
      & = 2 \cdot \frac{exp(\frac{-\epsilon (\ux{r} + k)}{2 S(\dataobs)})}{\unomalizer{\dataobs}} \cdot e^{\frac{\epsilon k}{2 S(\dataobs)}}\\
      & = 2 \cdot \frac{exp(\frac{-\epsilon (\ux{t})}{2 S(\dataobs)})}{\unomalizer{\dataobs}} \cdot e^{\frac{\epsilon k}{2 S(\dataobs)}}  (Apply \ \ux{t} = \ux{r} + k)\\
      & = 2 \cdot e^{\frac{\epsilon k}{2 S(\dataobs)}} \cdot \frac {exp\big(\frac{-\epsilon\cdot\hellinger(\bysinfer(\dataobs), t)}{2\cdot S(\dataobs)}\big)}{\nomalizer{\dataobs}} 
      (Apply\ \ux{r} = \hellinger(\bysinfer(\dataobs), r))\\
      & = 2 \cdot e^{\frac{\epsilon k}{2 S(\dataobs)}} \hexpmechPr{\dataobs}{z = t}       (Apply\ Definition\ \ref{def_smoo})\\
      & = 2 \cdot e^{\frac{\epsilon k}{2 S(\dataobs)}} \cdot \cardinality{\dataobs}{t} \cdot \hexpmechPr{\dataobs}{z = t} (Apply\ \cardinality{\dataobs}{t} = 1)\\
      & = 2 \cdot e^{\frac{\epsilon k}{2 S(\dataobs)}} \cdot \hexpmechPr{\dataobs}{\ux{z} = \ux{t}}       (Apply \ Lemma\ \ref{lem_score_pro_convert})\\
      & = 2 \cdot e^{\frac{\epsilon k}{2 S(\dataobs)}} \cdot \hexpmechPr{\dataobs}{\ux{z} = (\ux{r} + k)} (Apply \ \ux{t} = \ux{r} + k)\\
      & = e^{\ln 2} \cdot e^{\frac{\epsilon k}{2 S(\dataobs)}} \cdot \hexpmechPr{\dataobs}{\ux{z} = (\ux{r} + k)}\\
      & = e^{(\frac{\epsilon k}{2 S(\dataobs)} + \ln 2)} \cdot \hexpmechPr{\dataobs}{\ux{z} = (\ux{r} + k)}\\
      & = RHS.
      \end{split}
      \end{equation*} 

  \item {\boldmath $\ux{r} = 0, \ux{r} + k = 0$}: \todo{$r \in \mathcal{R}_1 \land t \in \mathcal{R}_1$}

        \begin{equation*}
        \begin{split}
        LHS 
        & = \hexpmechPr{\dataobs}{\ux{z} = \ux{r} + 0}\\
        & = \hexpmechPr{\dataobs}{\ux{z} = \ux{r} + k}\\
        & \leq e^{(\frac{\epsilon k}{2 S(x)} + \ln 2)} \cdot \hexpmechPr{\dataobs}{\ux{z} = ( \ux{r} + k)}\\
        & = RHS.
        \end{split}
        \end{equation*} 

\end{itemize}
Then, the proof is finished.

\end{proof}



\subsubsection{Dilation Property of $\hexpmech$}
\begin{lem}
\label{lem_dilation}
Given $\hexpmech(\dataobs)$ in Beta-binomial system, for any $r\in \candidateset$, let $1 - \frac{\delta'}{2} = |\candidateset|exp(\frac{\epsilon}{2 (1 - e ^ {\lambda})})$, the following \emph{dilation property} holds:
\begin{equation*}
\hexpmechPr{\dataobs}{\ux{z} = \ux{r}}
\leq
e^{\frac{\epsilon}{2}} \hexpmechPr{\dataobs}{\ux{z} = e^{\lambda} \ux{r}} + \frac{\delta'}{2}.
\end{equation*}
\end{lem}

\begin{proof} of Lemma \ref{lem_dilation}.

\todo{
The sensitivity is always greater than 0, and our utility function $-\hellinger(\bysinfer(\dataobs),z)$ is smaller than zero, i.e., $u(z,\dataobs) \leq 0$, we need to consider two cases where $\lambda < 0$, and $\lambda > 0$:\\
We set the $h(c) = Pr[u(\hexpmech(\dataobs)) = c] = \frac{exp(\frac{\epsilon c}{2 S(\dataobs)})}{NL(\dataobs)}$.\\
We first consider $\lambda < 0$. In this case, $1 < e ^ {\lambda}$, so the ratio $\frac{h(c)}{h(e^{\lambda}c)} = \frac{exp(\frac{\epsilon c}{2 S(\dataobs)})}{exp(\frac{\epsilon (c \cdot e^{\lambda})}{2 S(\dataobs)})}$ is at most $\frac{\epsilon}{2}$.\\
Next, we proof the dilation property for $\lambda > 0$, The ratio of $\frac{h(c)}{h(e^{\lambda}c)}$ is $\exp(\frac{\epsilon}{2} \cdot \frac{u(\hexpmech(\dataobs)) (1 - e^{\lambda})}{S(\dataobs)})$. Consider the event $G = \{ \hexpmech(\dataobs) : u(\hexpmech(\dataobs)) \leq \frac{S(\dataobs)}{(1 - e^{\lambda})}\}$. Under this event, the log-ratio above is at most $\frac{\epsilon}{2}$. The probability of $G$ under density $h(c)$ is $1 - \frac{\delta'}{2}$. Thus, the probability of a given event $z$ is at most $Pr[c \cap G] + Pr[\overline{G}] \leq e^{\frac{\epsilon}{2}} Pr[e^{\lambda}c \cap G] + \frac{\delta'}{2} \leq e^{\frac{\epsilon}{2}} Pr[e^{\lambda}c] + \frac{\delta'}{2}$.
}


  In Beta-binomial system, $\cardinality{\dataobs}{r}$ takes two possible values: $1$, $2$:
\begin{itemize}
  \item[-] $\cardinality{\dataobs}{r} = 1$, $\ux{r} = 0$.  $R^{\dataobs}_r$ contains only the correct posterior.
  \item[-] $\cardinality{\dataobs}{r} = 2$, $\ux{r} \neq 0$. $R^{\dataobs}_r$ contains 2 elements by the symmetry property of $\hellinger$.
\end{itemize}
 Let $\ux{t} = e^{\lambda} \ux{r}$.

  \begin{assert}
  \label{cardinality_eq}
  For any $r\in \candidateset$, $\cardinality{\dataobs}{r} = \cardinality{\dataobs}{t}$.
  \end{assert}

  \begin{proof} of Assertion \ref{cardinality_eq}.

  The proof is given by cases:
  \begin{itemize}
    \item {\boldmath$\ux{r} = 0$}:

        By $\ux{r} = 0$, $\cardinality{\dataobs}{r} = 1$.\\
        By $\ux{t} = e^{\lambda} \ux{r} = 0$, $\cardinality{\dataobs}{t} = 1$.

    \item {\boldmath$\ux{r} \neq 0$}:\\
        By $\ux{r} \neq 0$, $\cardinality{\dataobs}{r} = 2$.\\
        By $\ux{t} = e^{\lambda} \ux{r} \neq 0$, $\cardinality{\dataobs}{t} = 2$.
  \end{itemize}

  Then, in either case, $\cardinality{\dataobs}{r} = \cardinality{\dataobs}{t}$.

  \end{proof}

        To show:
        \begin{equation*}
        \hexpmechPr{\dataobs}{\ux{z} = \ux{r}}
        \leq
        e^{\frac{\epsilon}{2}} \hexpmechPr{\dataobs}{\ux{z} = e^{\lambda} \ux{r}} + \frac{\delta'}{2}.
        \end{equation*}

        We need to show 
        \begin{equation*}
        \hexpmechPr{\dataobs}{\ux{z} = \ux{r}}
        \leq
        e^{\frac{\epsilon}{2}} \hexpmechPr{\dataobs}{\ux{z} = e^{\lambda} \ux{r}},
        \end{equation*}
        holds with failure probability $\frac{\delta'}{2}$.

        By $\ux{t} = e^{\lambda} \ux{r}$, we need to show:
        \begin{equation*}
         \hexpmechPr{\dataobs}{\ux{z} = \ux{r}}
        \leq
        e^{\frac{\epsilon}{2}} \hexpmechPr{\dataobs}{\ux{z} = \ux{t}},
        \end{equation*}
        holds with failure probability $\frac{\delta'}{2}$.

        By Lemma \ref{lem_score_pro_convert}, we need to show:
        \begin{equation*}
        \cardinality{\dataobs}{r} \cdot \hexpmechPr{\dataobs}{z = r}
        \leq
        e^{\frac{\epsilon}{2}} \cardinality{\dataobs}{t} \cdot \hexpmechPr{\dataobs}{z = t}
        \end{equation*}
        holds with failure probability $\frac{\delta'}{2}$.

        By Assertion \ref{cardinality_eq}, we need to show:
        \begin{equation*}
        \hexpmechPr{\dataobs}{z = r}
        \leq
        e^{\frac{\epsilon}{2}} \hexpmechPr{\dataobs}{z = t},
        \end{equation*}
        holds with failure probability $\frac{\delta'}{2}$.

        By Definition \ref{def_smoo}, we need to show:
        \begin{equation*}
        \frac {exp\big(\frac{-\epsilon\cdot\hellinger(\bysinfer(\dataobs),r)}{2\cdot S(\dataobs)}\big)}{\nomalizer{\dataobs}}
        \leq
        e^{\frac{\epsilon}{2}} \frac {exp\big(\frac{-\epsilon\cdot\hellinger(\bysinfer(\dataobs),t)}{2\cdot S(\dataobs)}\big)}{\nomalizer{\dataobs}},
        \end{equation*}
        holds with failure probability $\frac{\delta'}{2}$.

        By simplification, $\ux{r} = \hellinger(\bysinfer(\dataobs),r)$, $\ux{t} = \hellinger(\bysinfer(\dataobs),t)$ and $\ux{t} = e^{\lambda} \ux{r}$, we need to show:
        \begin{equation*}
        \ux{r}(1 - e ^ {\lambda})
        \leq 
        S(\dataobs),
        \end{equation*}
        holds with failure probability $\frac{\delta'}{2}$.

        Then, prove it by two cases: $\lambda \leq 0$, and $\lambda > 0$:

      \begin{itemize}
        \item {\boldmath$\lambda < 0$}:

          \begin{equation*}
          \ux{r}(1 - e ^ {\lambda}) \leq 0 \leq S(\dataobs)
          \end{equation*}
          is always true.

        \item {\boldmath $\lambda > 0$}:

          To show:
          \begin{equation*}
          \ux{r}(1 - e ^ {\lambda})
          \leq 
          S(\dataobs),
          \end{equation*}
          holds with failure probability $\frac{\delta'}{2}$.

          We need to show:
          \begin{equation}
          \label{eq_dilation_case2}
          \ux{r}
          \leq 
          \frac{S(\dataobs)}{(1 - e ^ {\lambda})},
          \end{equation}         
          holds with failure probability $\frac{\delta'}{2}$.

          Based on the accuracy property of exponential mechanism:
          \begin{equation*}
          \begin{split}
          Pr[u(\expmech{}{x}{u}{\candidateset}) \leq c] 
          & \leq \frac{|\mathcal{R}|exp(\frac{\epsilon c}{2 GS})}{|\mathcal{R}_{OPT}| exp(\frac{\epsilon OPT_{u(\dataobs)}}{2 GS})},\\
          \end{split}
          \end{equation*}
          we derived the accuracy bound for $\hexpmech$:
          \begin{equation*}
          Pr[u(\hexpmech(\dataobs), \dataobs) \leq c] \leq |\candidateset|exp(\frac{\epsilon c}{2 S(\dataobs)}).
          \end{equation*}

          apply this bound to Equation. \ref{eq_dilation_case2}:
          \begin{equation*}
          \begin{split}
          Pr[\ux{r} \leq \frac{S(\dataobs)}{(1 - e ^ {\lambda})}] 
          & = |\candidateset|exp(\frac{\epsilon S(\dataobs)}{(1 - e ^ {\lambda})}/2 S(\dataobs))\\
          & = |\candidateset|exp(\frac{\epsilon}{2 (1 - e ^ {\lambda})}).\\
          \end{split}
          \end{equation*}

          By setting $|\candidateset|exp(\frac{\epsilon}{2 (1 - e ^ {\lambda})}) = 1 - \frac{\delta'}{2}$, 
          \begin{equation*}
          \begin{split}
          Pr[\ux{r} \leq \frac{S(\dataobs)}{(1 - e ^ {\lambda})}] = 1 - \frac{\delta'}{2}
          \end{split}
          \end{equation*}
          holds, i.e.,
          \begin{equation*}
          \ux{r}
          \leq 
          \frac{S(\dataobs)}{(1 - e ^ {\lambda})},
          \end{equation*}         
          holds with failure probability $\frac{\delta'}{2}$ is proved.

          Particularly, $|\candidateset| = n + 1$ in Beta-binomial system.

      \end{itemize}


\end{proof}

\subsubsection{$(\epsilon', \delta)-$Differential Privacy of $\hexpmech$}
\begin{lem}
\label{lem_hexpmech_privacy}
$\hexpmech$ is $(\epsilon', \delta)$-differential privacy in Beta-binomial system.
\end{lem}

\begin{proof} of Lemma \ref{lem_hexpmech_privacy}.

  By Definition \ref{def_epsilon_delta_dp}, to proof Lemma \ref{lem_hexpmech_privacy}, we need to prove:

  For any $\adj{\dataobs}{\dataobs'} \in \mathcal{X}$ and any beta distribution $r$:
  \begin{equation*}
  \hexpmechPr{\dataobs}{z = r} \leq e^{\epsilon'} \hexpmechPr{\dataobs'}{z = r} + \delta. 
  \end{equation*}

   In Beta-binomial system, $\cardinality{\dataobs}{r}$ takes two possible values: $1$, $2$, when considering data set $\dataobs$:
\begin{itemize}
  \item[-] $\cardinality{\dataobs}{r} = 1$, $\ux{r} = 0$.  $R^{\dataobs}_r$ contains only the correct posterior.
  \item[-] $\cardinality{\dataobs}{r} = 2$, $\ux{r} \neq 0$. $R^{\dataobs}_r$ contains 2 elements by the symmetry property of $\hellinger$.
\end{itemize}
 when considering data set $\dataobs'$:
  \begin{itemize}
    \item[*] $\cardinality{\dataobs'}{r} = 1$, $\uxadj{r} = 0$,  $R^{\dataobs'}_r$ contains only the correct posterior.
    \item[*] $\cardinality{\dataobs'}{r} = 2$, $\uxadj{r} \neq 0$. $R^{\dataobs'}_r$ contains 2 elements by the symmetry property of $\hellinger$.
  \end{itemize}

 
      Apply Lemma \ref{lem_score_pro_convert}:  
      \begin{equation*}
      \hexpmechPr{\dataobs}{z = r} = \frac{1}{\cardinality{\dataobs}{r}} \hexpmechPr{\dataobs}{\ux{z} = \ux{r}}.
      \end{equation*}

      Apply Lemma \ref{lem_sliding}:
      \begin{equation*}
      \begin{split}
      & \leq \frac{1}{\cardinality{\dataobs}{r}} \Big( e^{(\frac{\epsilon (\uxadj{r} - \ux{r})}{2 S(\dataobs)} + \ln 2)} \hexpmechPr{\dataobs}{\ux{z} = \ux{r} + (\uxadj{r} - \ux{r})} \Big)\\
      & = \frac{1}{\cardinality{\dataobs}{r}} \Big( e^{(\frac{\epsilon (\uxadj{r} - \ux{r})}{2 S(\dataobs)} + \ln 2)} \hexpmechPr{\dataobs}{\ux{z} = \uxadj{r} } \Big).
      \end{split}
      \end{equation*}

      Apply Lemma \ref{lem_dilation}:
      \begin{equation*}
      \begin{split}
      & \leq \frac{1}{\cardinality{\dataobs}{r}} \Big( e^{(\frac{\epsilon (\uxadj{r} - \ux{r})}{2 S(\dataobs)} + \ln 2)} 
      \big( e^{\frac{\epsilon}{2}}
      \hexpmechPr{\dataobs}{\ux{z} = \frac{S(\dataobs)}{S(\dataobs')} \ln(\frac{NL(\dataobs)}{NL(\dataobs')}) \uxadj{r} } \\
      & + 1 - |\candidateset|exp(\frac{\epsilon}{2 (1 - e ^ {\ln(\frac{S(\dataobs)}{S(\dataobs')} \ln(\frac{NL(\dataobs)}{NL(\dataobs')}))})}) 
      \big ) \Big).
      \end{split}
      \end{equation*}

      Let $\ux{t} = \frac{S(\dataobs)}{S(\dataobs')} \ln(\frac{NL(\dataobs)}{NL(\dataobs')}) \uxadj{r}$:      
      \begin{equation*}
      \begin{split}
      & = \frac{1}{\cardinality{\dataobs}{r}} \Big( e^{(\frac{\epsilon (\uxadj{r} - \ux{r})}{2 S(\dataobs)} + \ln 2)} 
      \big( e^{\frac{\epsilon}{2}}
      \hexpmechPr{\dataobs}{\ux{z} = \ux{t} } \\
      & + 1 - |\candidateset|exp(\frac{\epsilon}{2 (1 - e ^ {\ln(\frac{S(\dataobs)}{S(\dataobs')} \ln(\frac{NL(\dataobs)}{NL(\dataobs')}))})}) 
      \big ) \Big).
      \end{split}
      \end{equation*}

      Apply Lemma \ref{lem_score_pro_convert}:
      \begin{equation*}
      \begin{split}
      & = \frac{1}{\cardinality{\dataobs}{r}} \Big( e^{(\frac{\epsilon (\uxadj{r} - \ux{r})}{2 S(\dataobs)} + \ln 2)} 
      \big( e^{\frac{\epsilon}{2}}
      \cardinality{\dataobs}{t} \cdot \hexpmechPr{\dataobs}{z = t } \\
      & + 1 - |\candidateset|exp(\frac{\epsilon}{2 (1 - e ^ {\ln(\frac{S(\dataobs)}{S(\dataobs')} \ln(\frac{NL(\dataobs)}{NL(\dataobs')}))})}) 
      \big ) \Big).
      \end{split}
      \end{equation*}

      Apply Definition \ref{def_smoo} and $\ux{r} = \hellinger(\bysinfer(\dataobs),r)$ for $r \in \candidateset$:
      \begin{equation*}
      \begin{split}          
      & = \frac{1}{\cardinality{\dataobs}{r}} \Big( e^{(\frac{\epsilon (\uxadj{r} - \ux{r})}{2 S(\dataobs)} + \ln 2)} 
      \big( e^{\frac{\epsilon}{2}}
      \cardinality{\dataobs}{t}  \frac{exp(\frac{-\epsilon (\ux{t})}{2 S(\dataobs)})}{\unomalizer{\dataobs}}\\
      & + 1 - |\candidateset|exp(\frac{\epsilon}{2 (1 - e ^ {\ln(\frac{S(\dataobs)}{S(\dataobs')} \ln(\frac{NL(\dataobs)}{NL(\dataobs')}))})}) 
      \big ) \Big).
      \end{split}
      \end{equation*}

      Apply $\ux{t} = \frac{S(\dataobs)}{S(\dataobs')} \ln(\frac{NL(\dataobs)}{NL(\dataobs')}) \uxadj{r}$ and simplify:\\
      \begin{equation*}
      \begin{split}          
      & = \frac{1}{\cardinality{\dataobs}{r}} \Big( e^{(\frac{\epsilon (\uxadj{r} - \ux{r})}{2 S(\dataobs)} + \ln 2)} 
      \big( e^{\frac{\epsilon}{2}}
      \cardinality{\dataobs}{t} \frac{exp(\frac{-\epsilon (\uxadj{r})}{2 S(\dataobs')})}{\unomalizer{\dataobs'}}\\
      & + 1 - |\candidateset|exp(\frac{\epsilon}{2 (1 - e ^ {\ln(\frac{S(\dataobs)}{S(\dataobs')} \ln(\frac{NL(\dataobs)}{NL(\dataobs')}))})}) 
      \big ) \Big).
       \end{split}
      \end{equation*}

      Apply Definition \ref{def_smoo} and $\uxadj{r} = \hellinger(\bysinfer(\dataobs'),r)$ for $r \in \candidateset{\dataobs'}$:\\
      \begin{equation*}
      \begin{split}          
      & = \frac{1}{\cardinality{\dataobs}{r}} \Big( e^{(\frac{\epsilon (\uxadj{r} - \ux{r})}{2 S(\dataobs)} + \ln 2)} 
      \big( e^{\frac{\epsilon}{2}}
      \cardinality{\dataobs}{t}  \hexpmechPr{\dataobs'}{z = r} \\
      & + 1 - |\candidateset|exp(\frac{\epsilon}{2 (1 - e ^ {\ln(\frac{S(\dataobs)}{S(\dataobs')} \ln(\frac{NL(\dataobs)}{NL(\dataobs')}))})}) 
      \big ) \Big)\\
      & = \frac{\cardinality{\dataobs}{t}}{\cardinality{\dataobs}{r}} e^{(\frac{\epsilon (\uxadj{r} - \ux{r})}{2 S(\dataobs)} + \ln 2)} \cdot e^{\frac{\epsilon}{2}}
      \hexpmechPr{\dataobs'}{z = r} \\
      & + \frac{e^{(\frac{\epsilon (\uxadj{r} - \ux{r})}{2 S(\dataobs)} + \ln 2)}}{\cardinality{\dataobs}{r}}
      \Big( 1 - |\candidateset|exp(\frac{\epsilon}{2 \big( 1 - e ^ {\ln(\frac{S(\dataobs)}{S(\dataobs')} \ln(\frac{NL(\dataobs)}{NL(\dataobs')}))} \big) }) \Big).
      \end{split}
      \end{equation*}

      Then, $\hexpmech$ is $(\epsilon', \delta)-$differential privacy where $\epsilon' = \ln \Big( \frac{\cardinality{\dataobs}{t}}{\cardinality{\dataobs}{r}} e^{(\frac{\epsilon (\uxadj{r} - \ux{r})}{2 S(\dataobs)} + \ln 2)} \cdot e^{\frac{\epsilon}{2}} \Big)$, and
      $\delta = \frac{e^{(\frac{\epsilon (\uxadj{r} - \ux{r})}{2 S(\dataobs)} + \ln 2)}}{\cardinality{\dataobs}{r}}
      \Big( 1 - |\candidateset|exp(\frac{\epsilon}{2 \big(1 - e ^ {\ln(\frac{S(\dataobs)}{S(\dataobs')} \ln(\frac{NL(\dataobs)}{NL(\dataobs')}))}\big)}) \Big)$.


 Then the proof is given by cases:
  \begin{itemize}
    \item {\boldmath$\ux{r} \neq 0$, $\uxadj{r} \neq 0$}:

      By $\ux{r} \neq 0$, $\cardinality{\dataobs}{r} = 2$ under data set $\dataobs$.

      Apply $\uxadj{r} \neq 0 \implies \ux{t} \neq 0 \implies \cardinality{\dataobs}{t} = 2$.

      Apply  $\cardinality{\dataobs}{r} = 2$ and  $\cardinality{\dataobs}{t} = 2$:

      $\hexpmech$ is $(\epsilon', \delta)-$differential privacy where:
      $$
      \epsilon' = \ln \Big( e^{(\frac{\epsilon (\uxadj{r} - \ux{r})}{2 S(\dataobs)} + \ln 2)} \cdot e^{\frac{\epsilon}{2}} \Big),
      $$ and
      $$
      \delta = \frac{e^{(\frac{\epsilon (\uxadj{r} - \ux{r})}{2 S(\dataobs)} + \ln 2)}}{2}
      \Big( 1 - |\candidateset|exp(\frac{\epsilon}{2 \Big(1 - e ^ {\ln(\frac{S(\dataobs)}{S(\dataobs')} \ln(\frac{NL(\dataobs)}{NL(\dataobs')}))} \Big)}) \Big).
      $$ 

    \item {\boldmath$\ux{r} = 0$, $\uxadj{r} \neq 0$}:

      By $\ux{r} = 0$, $\cardinality{\dataobs}{r} = 1$ under data set $\dataobs$.
      
      Apply $\uxadj{r} \neq 0 \implies \ux{t} \neq 0 \implies \cardinality{\dataobs}{t} = 2$.

      Apply  $\cardinality{\dataobs}{r} = 1$ and  $\cardinality{\dataobs}{t} = 2$:

      $\hexpmech$ is $(\epsilon', \delta)-$differential privacy where:
      $$
      \epsilon' = \ln \Big( 2 \cdot e^{(\frac{\epsilon (\uxadj{r} - \ux{r})}{2 S(\dataobs)} + \ln 2)} \cdot e^{\frac{\epsilon}{2}} \Big),
      $$ and
      $$
      \delta = e^{(\frac{\epsilon (\uxadj{r} - \ux{r})}{2 S(\dataobs)} + \ln 2)}
      \Big( 1 - |\candidateset|exp(\frac{\epsilon}{2 \Big( 1 - e ^ {\ln(\frac{S(\dataobs)}{S(\dataobs')} \ln(\frac{NL(\dataobs)}{NL(\dataobs')}))}\Big) }) \Big).
      $$ 
      

    \item {\boldmath$\ux{r} \neq 0$, $\uxadj{r} = 0$}:

      By $\ux{r} \neq 0$, $\cardinality{\dataobs}{r} = 2$ under data set $\dataobs$.
      
      Apply $\uxadj{r} = 0 \implies \ux{t} = 0 \implies \cardinality{\dataobs}{t} = 1$.

      Apply  $\cardinality{\dataobs}{r} = 2$ and  $\cardinality{\dataobs}{t} = 2$:

      $\hexpmech$ is $(\epsilon', \delta)-$differential privacy where:
      $$
      \epsilon' = \ln \Big( \frac{1}{2} e^{(\frac{\epsilon (\uxadj{r} - \ux{r})}{2 S(\dataobs)} + \ln 2)} \cdot e^{\frac{\epsilon}{2}} \Big),
      $$ and
      $$
      \delta = \frac{e^{(\frac{\epsilon (\uxadj{r} - \ux{r})}{2 S(\dataobs)} + \ln 2)}}{2}
      \Big( 1 - |\candidateset|exp(\frac{\epsilon}{2 \Big( 1 - e ^ {\ln(\frac{S(\dataobs)}{S(\dataobs')} \ln(\frac{NL(\dataobs)}{NL(\dataobs')}))} \Big)}) \Big).
      $$ 

    \item {\boldmath$\ux{r} = 0$, $\uxadj{r} = 0$}:

      By $\ux{r} \neq 0$, $\cardinality{\dataobs}{r} = 1$ under data set $\dataobs$.
      
      Apply $\uxadj{r} = 0 \implies \ux{t} = 0 \implies \cardinality{\dataobs}{t} = 1$.

      Apply  $\cardinality{\dataobs}{r} = 1$ and  $\cardinality{\dataobs}{t} = 1$:

      $\hexpmech$ is $(\epsilon', \delta)-$differential privacy where:
      $$
      \epsilon' = \ln \Big( e^{(\frac{\epsilon (\uxadj{r} - \ux{r})}{2 S(\dataobs)} + \ln 2)} \cdot e^{\frac{\epsilon}{2}} \Big),
      $$ and
      $$
      \delta = e^{(\frac{\epsilon (\uxadj{r} - \ux{r})}{2 S(\dataobs)} + \ln 2)}
      \Big( 1 - |\candidateset|exp(\frac{\epsilon}{2 \Big( 1 - e ^ {\ln(\frac{S(\dataobs)}{S(\dataobs')} \ln(\frac{NL(\dataobs)}{NL(\dataobs')}))} \Big)}) \Big).
      $$ 

  \end{itemize}

  
      Then, $\hexpmech$ is $(\epsilon', \delta)-$differential privacy, by setting $\epsilon $ and $\delta$ as the maximum value from above.
      i.e., under the case {\boldmath$\ux{r} = 0$, $\uxadj{r} \neq 0$}, 
      $
      \epsilon' = \ln (2 \cdot e^{(\frac{\epsilon (\uxadj{r} - \ux{r})}{2 S(\dataobs)} + \ln 2)} \cdot e^{\frac{\epsilon}{2}})
      $, and
      $
      \delta = e^{(\frac{\epsilon (\uxadj{r} - \ux{r})}{2 S(\dataobs)} + \ln 2)}
      \Big( 1 - |\candidateset|exp(\frac{\epsilon}{2 \big(1 - e ^ {\ln(\frac{S(\dataobs)}{S(\dataobs')} \ln(\frac{NL(\dataobs)}{NL(\dataobs')}))} \big)}) \Big).
      $\\

\end{proof}

\bibliographystyle{plain}
\bibliography{bayesian.bib}

\end{document}

