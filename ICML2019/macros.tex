\usepackage{natbib}
\usepackage{amsmath}
\usepackage{amsthm}
\usepackage{mathtools}
\usepackage{mdframed}
\usepackage{subfigure}
\usepackage{booktabs}
% \usepackage{hyperref}
\usepackage{subfigure}
\usepackage{siunitx} % Provides the \SI{}{} and \si{} command for typesetting SI units
\usepackage{graphicx} % Required for the inclusion of images
% \usepackage{natbib} % Required to change bibliography style to APA
\usepackage{datetime}
\usepackage{lscape}
\usepackage{geometry}
\usepackage{algorithm}
\usepackage{algorithmic}
\usepackage{xspace}
\usepackage[english]{babel} % English language/hyphenation
\usepackage{proof}
\usepackage{booktabs} % Top and bottom rules for tables
\usepackage[colorlinks, allcolors = blue,]{hyperref}
\usepackage{accents}
\usepackage{amsfonts}
\usepackage{stmaryrd}
\usepackage{amsmath,amsthm,amssymb,latexsym} 
\usepackage{microtype}
\usepackage{graphicx}
\usepackage{subfigure}
\usepackage{booktabs} % for professional tables

% hyperref makes hyperlinks in the resulting PDF.
% If your build breaks (sometimes temporarily if a hyperlink spans a page)
% please comment out the following usepackage line and replace
% \usepackage{icml2019} with \usepackage[nohyperref]{icml2019} above.
\usepackage{hyperref}

% Attempt to make hyperref and algorithmic work together better:
\newcommand{\theHalgorithm}{\arabic{algorithm}}

% Use the following line for the initial blind version submitted for review:
\usepackage{icml2019}


\usepackage{authblk}
%\usepackage[usenames, dvipsnames]{color}
%\setlength\parindent{0pt}
\newtheorem{definition}{Definition}
\usepackage{cancel}
\usepackage[normalem]{ulem}
\newcommand{\dataobs}{\textbf{x}}
\newcommand{\adj}[2]{\textbf{adj}(#1,#2)}
\newcommand{\candidateset}{\mathcal{R}_{\textup{post}}}
\newcommand{\bprior}{\boldsymbol{\beta}_{\textup{prior}}}
\newcommand{\bysinfer}{\mathsf{Infer}}
\newcommand{\betad}{\mathsf{Beta}}
\newcommand{\betaf}{\textup{B}}
\newcommand{\mbetaf}{\boldsymbol{\textup{B}}}
\newcommand{\vtheta}{\boldsymbol{\theta}}
\newcommand{\valpha}{\boldsymbol{\alpha}}
\newcommand{\vbeta}{\boldsymbol{\beta}}
\newcommand{\lapmech}{\mathsf{lapMech}}
\newcommand{\ilapmech}{\mathsf{ilapMech}}
\newcommand{\binomial}[2]{\mathsf{Bin}(#1, #2)}
\newcommand{\multinomial}[2]{\mathsf{Mult}(#1, #2)}
\newcommand{\expmech}{\mathsf{expMech}}
\newcommand{\hexpmech}{\mathsf{expMech}^{smoo}}
\newcommand{\lexpmech}{\mathsf{expMech}^{local}}
\newcommand{\hexpmechd}{\mathsf{expMech}^{D}_{\hellinger}}
\newcommand{\privinfer}{\mathsf{PrivInfer}}
\newcommand{\hlg}{\mathsf{H}}
\newcommand{\dirichlet}[1]{\mathsf{Dir}(#1)}
\newcommand{\alphas}{\boldsymbol{\alpha}}
\newcommand{\xis}{\boldsymbol{\xi}}
\newcommand{\iverson}[1]{[#1]}
\newcommand{\datauni}{\mathcal{X}}
\newcommand{\hellinger}{\mathcal{H}}
\newcommand{\ux}[1]{u(\textbf{x}, {#1})}
\newcommand{\uxadj}[1]{u(\textbf{x}', {#1})}
\newcommand{\cardinality}[2]{\mathcal{C}^{#1}_{#2}}
\newcommand{\range}{\mathcal{O}}
\newcommand{\nomalizer}[1]{\sum\limits_{r'\in \mathcal{R}_{\textup{post}}} \exp \big(\frac{-\epsilon\cdot \mathcal{H} (\mathsf{BI}(#1),r')}{4 \cdot S(#1)}\big)}

\newcommand{\unomalizer}[1]{\sum\limits_{r'\in \mathcal{R}_{\textup{post}}} \exp \big(\frac{-\epsilon\cdot u(#1, r')}{4 \cdot S(#1)}\big)}


\newcommand{\hexpmechPr}[2]{\underset{z \thicksim \hexpmech(#1)}{Pr}[#2]}
\newcommand{\lapmechPr}[2]{\underset{z \thicksim \lapmech(#1)}{Pr}[#2]}
\newcommand{\ilapmechPr}[2]{\underset{z \thicksim \ilapmech(#1)}{Pr}[#2]}

% \newcommand{\because}{\mathsf{because}}
% \newcommand{\therefore}{\mathsf{therefore}}
%\newcommand{\theHalgorithm}{\arabic{algorithm}}
% \theoremstyle{plain}
\newtheorem{thm}{Theorem}[section]

\newtheorem{lem}{Lemma}[section]
% \newtheorem{prop}[thm]{Proposition}
\newtheorem{assert}{Assertion}[lem]

% \newcommand{mech}{Mechanism}[algorithmic]

% \newtheorem*{cor}{Corollary}
% \theoremstyle{definition}
% % \newtheorem{definition}{Definition}[section]
% \newtheorem{defn}{Definition}[section]
% \newtheorem{conj}{Conjecture}[section]
% \newtheorem{exmp}{Example}[section]

% \theoremstyle{remark}
% \newtheorem*{rem}{Remark}
% \newtheorem*{note}{Note}
\newcommand{\lap}[2]{\mathsf{Lap}(#1, #2)}
\newcommand{\todo}[1]{{\footnotesize \color{red}\textbf{[[ #1 ]]}}}